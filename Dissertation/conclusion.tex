\chapter*{Заключение}                       % Заголовок
\addcontentsline{toc}{chapter}{Заключение}  % Добавляем его в оглавление

%% Согласно ГОСТ Р 7.0.11-2011:
%% 5.3.3 В заключении диссертации излагают итоги выполненного исследования, рекомендации, перспективы дальнейшей разработки темы.
%% 9.2.3 В заключении автореферата диссертации излагают итоги данного исследования, рекомендации и перспективы дальнейшей разработки темы.
%% Поэтому имеет смысл сделать эту часть общей и загрузить из одного файла в автореферат и в диссертацию:

В диссертационном исследовании получены следующие результаты.
%% Согласно ГОСТ Р 7.0.11-2011:
%% 5.3.3 В заключении диссертации излагают итоги выполненного исследования, рекомендации, перспективы дальнейшей разработки темы.
%% 9.2.3 В заключении автореферата диссертации излагают итоги данного исследования, рекомендации и перспективы дальнейшей разработки темы.
\begin{enumerate}
  \item Построена стохастическая модель, описывающая предложенный игровой конфликт двух игроков, управляющих блужданием фишки на конечной квадратной решетке. 
  \item Разработаны методы для расчета статистических характеристик игрового процесса при фиксированных заданных стратегиях игроков,
        таких как среднее время игры, распределение времен игры, распределение вероятностей наблюдения фишки в состояниях конечной решетки. 
  \item Выдвинута гипотеза об оптимальном среднем времени для двух случаев игры против стратегии случайного равновероятного выбора,
        и установлена оптимальность данных значений для малых размеров игрового поля, а также найден класс стратегий, достигающих предполагаемой оптимальной оценки.
  \item Разработано мобильное приложение Random Walk Game, реализующее игровую механику посредством сети интернет с использованием
        созданного веб-сервера по обработке и хранению результатов игр участников. Дополнительно разработан веб-сайт для
        отображения статистической информации по результатам игр участников в режиме реального времени.
        Приложение опубликовано в открытом доступе на двух маркетплейсах для платформ Android и iOS.
  \item Проведен масштабный эксперимент с применением мобильных и интернет-технологий, привлекший более 100 участников
        и позволивший собрать более 1500 игр Random Walk Game. 
  \item Подтверждено соответствие эксперимента предложенной модели для трех случаев игры на основе сравнения
        распределений и соответствующих средних времен игры. 
  \item Установлены возникающие особенности синхронизации игроков при длительных играх, связанные со снижением концентрации игроков.
        Длительные игры утомляют игроков, что снижает способность человека генерировать чисто случайную последовательность выборов.
  \item Выявлены отличия стратегий игроков от предполагаемых оптимальных стратегий
        на основе сравнения статистических свойств траекторий и стратегий участников для случаев игры против стратегии равновероятного случайного выбора.
\end{enumerate}


Дальнейшие исследования могут быть связаны с поиском квазистационарных состояний, возникающих в поглощающих Марковских цепях. Отдельный интерес представляет измерение уровня синхронизации между игроками, изменяющегося в течение игры. Необходимой составляющей для проведения такого типа анализа является возможность получения большего числа длительных игр между игроками, что позволит выявить причины возникновения корреляции между решениями участников на каждом ходу. Построенная в данной работе стохастическая модель игры и реализация мобильного могут быть использованы другими исследователями теории игр как отправная точка для построения игр, обладающих свойством описания решений участников популяции и воспроизведения их поведения в модели. 

В заключение автор выражает благодарность и большую признательность профессору Денисову~С.\,В. за поддержку, помощь, обсуждение результатов, прямое участие в игровом процессе, за предложенную идею, за помощь в разработке дизайна игры и~научное руководство. Также автор благодарит научного руководителя Иванченко~М.\,В. за поддержку, помощь в организации процесса подготовки диссертации, проверку текста диссертации, конструктивные замечания и рекомендации, Тихомирова~С.\,Н. за помощь в~анализе игр, помощь в разработке мобильного приложения, помощь в организации мероприятий и поиске участников, Карчкова~Д.\,А. за предоставленный экскурс в разработку кроссплатформенных мобильных приложений, Забурдаева~В.\,Ю. за~постановку задач в области хемотаксиса и помощь в применении аналитического аппарата при исследовании хемотаксиса, Полевую~С.А. за разработанные когнитивные тесты, Кондакову~Е.В. за подготовку выборки результатов прохождения когнитивных тестов и авторов шаблона *Russian-Phd-LaTeX-Dissertation-Template* за~помощь в оформлении диссертации. Автор также благодарит всех участников игрового эксперимента и~всех, кто сделал настоящую работу автора возможной.
