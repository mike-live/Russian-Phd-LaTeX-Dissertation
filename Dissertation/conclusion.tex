\chapter*{Заключение}                       % Заголовок
\addcontentsline{toc}{chapter}{Заключение}  % Добавляем его в оглавление

%% Согласно ГОСТ Р 7.0.11-2011:
%% 5.3.3 В заключении диссертации излагают итоги выполненного исследования, рекомендации, перспективы дальнейшей разработки темы.
%% 9.2.3 В заключении автореферата диссертации излагают итоги данного исследования, рекомендации и перспективы дальнейшей разработки темы.
%% Поэтому имеет смысл сделать эту часть общей и загрузить из одного файла в автореферат и в диссертацию:

В диссертационном исследовании получены следующие результаты.
%% Согласно ГОСТ Р 7.0.11-2011:
%% 5.3.3 В заключении диссертации излагают итоги выполненного исследования, рекомендации, перспективы дальнейшей разработки темы.
%% 9.2.3 В заключении автореферата диссертации излагают итоги данного исследования, рекомендации и перспективы дальнейшей разработки темы.
\begin{enumerate}
  \item Построена стохастическая модель генерации степенных распределений длительностей нахождения системы в одном из двух состояний за счет дробового шума. Получен способ оценки распределения длительностей для модели. Найдены параметры для генерации двух чередующихся режимов: степенное распределение и экспоненциальное распределение длительностей.
  \item Получена аналитическая форма средней скорости колонии бактерий в случае паттерна движения с двумя чередующимися углами. Проведенный численный эксперимент позволил подтвердить корректность формулы при малом химическом градиенте, а также продемонстрировать наличие отклонений при большом градиенте. 
  \item Получена оценка параметров химической чувствительности бактерий и коэффициента диффузии при их движении в условиях нелинейной радиальной концентрации химического вещества с применением численного эксперимента.
  \item Построена стохастическая модель, описывающая предложенный игровой конфликт двух игроков, управляющих блужданием фишки на конечной квадратной решетке. 
  \item Разработаны методы для расчета статистических характеристик игрового процесса при фиксированных заданных стратегиях игроков, таких как среднее время игры, распределение времен игры, распределение вероятностей наблюдения фишки в состояниях конечной решетки. 
  \item Вычислены оптимальные средние времена для трех случаев игры, предложены классы оптимальных стратегий и визуализированы конкретные стратегии. Предложен подход для нахождения оптимальных стратегий при произвольной стратегии оппонента, а также при оптимальной стратегии.
  \item Разработано мобильное приложение Random Walk Game, реализующее игровую механику посредством сети интернет с использованием созданного веб-сервера по обработке и хранению результатов игр участников. Дополнительно разработан веб-сайт для отображения статистической информации по результатам игр участников в режиме реального времени. Приложение опубликовано в открытом доступе на двух маркетплейсах для платформ Android и iOS.
  \item Проведен масштабный эксперимент с применением мобильных и интернет-технологий, привлекший более 100 участников и позволивший собрать более 1500 игр Random Walk Game. 
  \item Подтверждено соответствие эксперимента предложенной модели для трех случаев игры на основе сравнения распределений и соответствующих средних времен игры. 
  \item Установлены возникающие особенности синхронизации игроков при длительных играх, связанные со снижением концентрации игроков. Длительные игры утомляют игроков, что снижает способность человека генерировать чисто случайную последовательность выборов.
  \item Выявлены отличия стратегий игроков от оптимальных стратегий на основе сравнения статистических свойств траекторий и стратегий участников для случаев игры против стратегии равновероятного случайного выбора.
\end{enumerate}


Дальнейшие исследования могут быть связаны с поиском квазистационарных состояний, возникающих в поглощающих Марковских цепях. Отдельный интерес представляет измерение уровня синхронизации между игроками, изменяющегося в течение игры. Необходимой составляющей для проведения такого типа анализа является возможность получения большего числа длительных игр между игроками, что позволит выявить причины возникновения корреляции между решениями участников на каждом ходу. Построенная в данной работе стохастическая модель игры и реализация мобильного могут быть использованы другими исследователями теории игр как отправная точка для построения игр, обладающих свойством описания решений участников популяции и воспроизведения их поведения в модели. 

В заключение автор выражает благодарность научному руководителю Иванченко~М.\,В. за руководство на разных этапах работы, поддержку, помощь в организации процесса подготовки диссертации, проверку текста диссертации, конструктивные замечания и рекомендации, и профессору Денисову~С.\,В. за руководство на разных этапах работы, обсуждение результатов, прямое участие в игровом процессе, за идею игры, помощь в разработке дизайна игры. Также автор благодарит Тихомирова~С.\,Н. за помощь в~анализе игр, помощь в разработке мобильного приложения, помощь в организации мероприятий и поиске участников, Карчкова~Д.\,А. за предоставленный экскурс в разработку кроссплатформенных мобильных приложений, Забурдаева~В.\,Ю. за~постановку задач в области хемотаксиса и помощь в применении аналитического аппарата при исследовании хемотаксиса, Полевую~С.\,А. за разработанные когнитивные тесты, Кондакову~Е.\,В. за подготовку выборки результатов прохождения когнитивных тестов, авторов шаблона *Russian-Phd-LaTeX-Dissertation-Template* за~помощь в оформлении диссертации и Пивневу~Н.\,А. за профессиональную корректуру текста диссертации. Автор также благодарит всех участников игрового эксперимента и~всех, кто сделал настоящую работу автора возможной.
