\chapter*{Словарь терминов}             % Заголовок
\addcontentsline{toc}{chapter}{Словарь терминов}  % Добавляем его в оглавление

\textbf{Случайный процесс} : Семейство случайных величин, индексированных некоторым параметром, играющим роль времени.

\textbf{Случайное блуждание} : Случайный процесс, описывающий путь, состоящий из последовательности случайных шагов в математическом пространстве.

\textbf{Цепь Маркова} : Последовательность случайных событий с конечным или счётным числом исходов, где вероятность наступления каждого события зависит только от состояния, достигнутого в предыдущем событии

\textbf{Двудольный граф} : Граф, множество вершин которого можно разбить на две части таким образом, 
что каждое ребро графа соединяет вершину из одной части с какой-то вершиной другой части, то есть не существует рёбер между вершинами одной и той же части графа.

\textbf{Антагонистические игры} : Некооперативная игра, в которой участвуют два или более игроков, выигрыши которых противоположны.

\textbf{Некооперативная игра} : Математическая модель взаимодействия нескольких сторон (игроков), в процессе которого они не могут формировать коалиции и координировать свои действия.

\textbf{Матрица Теплица} : Матрица, в которой на всех диагоналях, параллельных главной, стоят равные элементы.

