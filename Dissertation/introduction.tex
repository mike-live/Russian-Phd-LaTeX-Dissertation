\chapter*{Введение}                         % Заголовок
\addcontentsline{toc}{chapter}{Введение}    % Добавляем его в оглавление

\newcommand{\actuality}{\textbf\actualityTXT}
\newcommand{\progress}{}
\newcommand{\aim}{{\textbf\aimTXT}}
\newcommand{\tasks}{\textbf{\tasksTXT}}
\newcommand{\novelty}{\textbf{\noveltyTXT}}
\newcommand{\influence}{\textbf{\influenceTXT}}
\newcommand{\methods}{\textbf{\methodsTXT}}
\newcommand{\defpositions}{\textbf{\defpositionsTXT}}
\newcommand{\reliability}{\textbf{\reliabilityTXT}}
\newcommand{\probation}{\textbf{\probationTXT}}
\newcommand{\contribution}{\textbf{\contributionTXT}}
\newcommand{\publications}{\textbf{\publicationsTXT}}


{\actuality} <<Игровые>> взаимодействия, в которых каждый участник руководствуется принципами максимизации выигрыша в некоторой игре -- 
либо своего личного, либо целого коллектива («популяции») – существенно отличны от обычных физических взаимодействий. 
Дополненные механизмами обучения, оценки и адаптации, игровые взаимодействия определяют принципиально новый тип динамики или <<эволюции>> [1]. 
К настоящему времени стало понятно, что область применения игровой динамики гораздо шире социологии и экологии, и что соответствующие 
методы могут быть использованы для моделирования динамики финансовых рынков [2] и интерпретации процессов Бозе-Эйнштейн конденсации в открытых квантовых системах [3]. 

Объект исследования данной работы — процессы случайных блужданий, обусловленные конфликтным взаимодействием игроков. 
Модельный процесс представляет собой игру, в которой игроки управляют перемещением точки (индикатор положения) на квадратной решетке, 
делая независимый выбор одной из двух возможных стратегий (<<ход>>) на каждом шаге игры. Информация о возможных перемещениях точки 
(определяемых совместным выбором), открыта для обоих игроков и организована в виде матрицы. Целью первого игрока является максимально 
долгое удержание точки внутри ограниченной области, а второго -- максимально быстрое достижение ею поглощающей границы. 
Результатом игры («платежом») является время поглощения. 

Игры такого типа были определены в некоторых работах по теории игр и принятия решений еще в 1950-1960х годах [4,5]. 
Однако, в силу вычислительной сложности задачи, количественные результаты были получены только для очень простых моделей 
с тремя-пятью состояниями [6], что не позволяет делать заключения об асимптотических характеристиках процесса и использовать 
для их анализа статистический подход. Существует тесная связь между «играми на выживание» и процессами случайных блужданий, 
например, классическая задача о разорении игрока допускает прямую интерпретацию как процесс случайного блуждания на конечном интервале [6]. 

С другой стороны, проблема случайных блужданий в ограниченной области и такие вопросы, как оценка среднего времени достижения границы 
(или любого другого определённого региона). [7], в настоящее время испытывает очередную волну интереса, вызванную перспективами 
приложения этого подхода к проблемам молекулярной биологии, химической кинетики, и экологии [8-9]. 

Идея данной работы состоит в том, что игровые <<блуждания>> в конечных пространственных областях, ограниченных поглощающими границами 
(достижение границы является выигрышем одного из игроков и, соответственно, проигрышем его оппонента), 
могут быть исследованы и квантифицированы, используя методологию теории случайных блужданий. Более конкретно, 
предлагается исследовать распределение времени достижения границы (<<времени игры>>). Очевидно, что такое распределение 
будет существенно отличаться от распределения времени достижения границ для стандартных моделей случайных блужданий 
в той же ограниченной области [8]. Одна из задач -- построение стохастической модели, которая бы воспроизводила это распределение. 
В работе проводится сравнение теоретических результатов с экспериментальными данными, собранными c помощью мобильного приложения. 
Следует отметить, что использование методик таких экспериментов (в областях далеких от социологии, где, главным образом, эта методика локализована) 
является одним из новейших трендов в области прикладной математики и математического моделирования; см., например [10-12].
Данная работа предполагает проведение масштабного «полевого эксперимента» с участием реальных игроков. 
Таким образом, область исследований лежит на стыке трех областей: теории случайных блужданий, теории игр и принятия решений 
и технологии социальных экспериментов с использованием мобильных приложений. 

В соответствии с паспортом специальности 05.13.18 <<Математическое моделирование,
численные методы и комплексы программ>>, данная диссертация
относится к следующим областям исследований: <<3. Разработка, обоснование
и тестирование эффективных вычислительных методов с применением современных
компьютерных технологий>>, <<4. Реализация эффективных численных методов и алгоритмов в виде
комплексов проблемно-ориентированных программ для проведения
вычислительного эксперимента>> <<5. Комплексные исследования научных и
технических проблем с применением современной технологии математического
моделирования и вычислительного эксперимента>>, <<7. Разработка новых математических
методов и алгоритмов интерпретации натурного эксперимента на
основе его математической модели>>, <<8. Разработка систем компьютерного и имитационного моделирования>>.

\ifsynopsis
Этот абзац появляется только в~автореферате.
Для формирования блоков, которые будут обрабатываться только в~автореферате,
заведена проверка условия \verb!\!\verb!ifsynopsis!.
Значение условия задаётся в~основном файле документа (\verb!synopsis.tex! для
автореферата).
\else
Этот абзац появляется только в~диссертации.
Через проверку условия \verb!\!\verb!ifsynopsis!, задаваемого в~основном файле
документа (\verb!dissertation.tex! для диссертации), можно сделать новую
команду, обеспечивающую появление цитаты в~диссертации, но~не~в~автореферате.
\fi

% {\progress}
% Этот раздел должен быть отдельным структурным элементом по
% ГОСТ, но он, как правило, включается в описание актуальности
% темы. Нужен он отдельным структурынм элемементом или нет ---
% смотрите другие диссертации вашего совета, скорее всего не нужен.

{\aim} данной работы является получение информации о статистических
характеристиках процесса блужданий в ограниченной двухмерной области, индуцированного игровым
взаимодействием двух игроков-оппонентов и построение стохастической модели, способной воспроизвести эти характеристики. 

Для~достижения поставленной цели необходимо было решить следующие {\tasks} диссертационного исследования:
\begin{enumerate}[beginpenalty=10000] % https://tex.stackexchange.com/a/476052/104425
    \item Разработка стохастической модели блужданий на плоскости, управляемых
    игровым конфликтом, и получение характеристик игрового процесса.
    \item Разработка и имплементация алгоритмов для моделирования
    игровой пространственной динамики, используя стохастическую
    модель, получение информации о статистических характеристиках
    модельного процесса, таких как время достижения границы и его
    распределение.
    \item Реализация масштабной серии экспериментов с участием
    реальных игроков и получение статистически значимого массива
    данных.
    \item Исследование взаимосвязи результатов, полученных в ходе
    теоретического анализа, численного моделирования, и эксперимента, а
    также их совместная интерпретация.
\end{enumerate}


{\novelty}
В работе получены следующие новые научные результаты:
\begin{enumerate}[beginpenalty=10000] % https://tex.stackexchange.com/a/476052/104425
  \item Впервые предложен игровой конфликт двух игроков, управляющих блужданием фишки на плоскости, реализованный в виде мобильного приложения.
        Новизна подхода заключается в применении интернет-технологий для реализации игры одновременно учитывающей как 
        возможность создания игрового взаимодействия между игроками, так и процесса случайного блуждания.
  \item Впервые разработана стохастическая модель предложенной игровой динамики, позволяющая получить характеристики игрового процесса 
        и воспроизвести результаты масштабного эксперимента игр полученных реальными игроками. Новизна подхода состоит в возможности 
        исследования модельного процесса случайного блуждания, позволяющего воспроизвести экспериментально полученные характеристики,
        а также провести точное сравнение модели с полевым экспериментом.
  \item Разработаны и реализованы численные методы для расчета статистических характеристик игрового процесса при фиксированных заданных стратегиях игроков,
        таких как среднее время игры, распределение времен игры, распределение вероятностей наблюдения фишки в состояниях конечной решетки. 
  \item Выдвинута гипотеза об оптимальном среднем времени оптимальных стратегий для двух случаев игры против стратегии случайного равновероятного выбора, продемонстрирована
        оптимальность данных значений для малых размеров игрового поля и предложены классы стратегий, достигающие предполагаемой оптимальной оценки.
\end{enumerate}

%{\influence} \ldots

{\methods} 
В работе используются методы математического моделирования, теории игр, теории вероятностей, математической статистики, 
теории Марковских цепей, теории случайных блужданий, математического программирования, численного моделирования. 
Дополнительно используется подход к проведению масштабного полевого эксперимента с применением интернет-технологий и мобильных приложений.


{\defpositions}
\begin{enumerate}[beginpenalty=10000] % https://tex.stackexchange.com/a/476052/104425
  \item Построена стохастическая модель, описывающая игровой конфликт двух игроков, управляющих блужданием фишки на конечной квадратной решетке.
  \item Созданы методы для расчета статистических характеристик игрового процесса при фиксированных заданных стратегиях игроков,
        таких как среднее время игры, распределение времен игры, распределение вероятностей наблюдения фишки в состояниях конечной решетки. 
  \item Выдвинута гипотеза об оптимальном среднем времени для двух случаев игры против стратегии случайного равновероятного выбора,
        и установлена оптимальность данных значений для малых размеров игрового поля, а также найден класс стратегий, достигающих предполагаемой оптимальной оценки.
  \item Обнаружено соответствие эксперимента предложенной модели и установлены возникающие особенности синхронизации игроков, 
        а также выявлены отличия стратегий игроков от предполагаемых оптимальных стратегий
        на основе сравнения статистических свойств траекторий и стратегий участников масштабного эксперимента, 
        проведенного с применением мобильных и интернет-технологий.
  
\end{enumerate}

{\reliability} полученных результатов, научных положений и выводов,
полученных в диссертации, обеспечивается корректным обоснованием
постановок задач, точной формулировкой критериев, подтверждается результатами
вычислительных экспериментов по использованию предложенных в
диссертации методов и алгоритмов, сравнением полученных результатов с
проведёнными ранее исследованиями и перекрестной проверкой с применением трех различных методов.
Результаты находятся в соответствии с результатами, полученными другими авторами.

{\probation}
Основные результаты диссертационного исследования докладывались~на следующих научных конференциях и фестивалях:
\begin{itemize}
    \item XXVI научная конференция по радиофизике, посвященная 120-летию со дня рождения М.Т. Греховой. 2022, ННГУ им. Н.И. Лобачевского,
    Нижний Новгород.
    \item Всероссийский фестиваль молодежных инноваций Иннофест. 2020, ННГУ им. Н.И. Лобачевского,
    Нижний Новгород.
\end{itemize}


{\contribution} Решение задач диссертационного исследования, разработка стохастической модели, описывающей игровой конфликт двух игроков,
построение и реализация методов для расчета статистических характеристик, разработка гипотезы об оптимальном среднем времени игры
для двух случаев игры против стратегии случайного равновероятного выбора принадлежат автору лично.
Разработка мобильного приложения, разработка серверной части платформы, организация масштабного эксперимента 
и сопутствующих мероприятий выполнены при непосредственном активном участии автора.
Автор принимал прямое участие в постановке задач и анализе полученных результатов, а также в подготовке публикаций
в научных журналах и докладов на тематических конференциях.

\ifnumequal{\value{bibliosel}}{0}
{%%% Встроенная реализация с загрузкой файла через движок bibtex8. (При желании, внутри можно использовать обычные ссылки, наподобие `\cite{vakbib1,vakbib2}`).
    {\publications} Основные результаты по теме диссертации изложены
    в~2~печатных изданиях,
    0 из которых изданы в журналах, рекомендованных ВАК,
    1 "--- в тезисах докладов.
}%
{%%% Реализация пакетом biblatex через движок biber
    \begin{refsection}[bl-author, bl-registered]
        % Это refsection=1.
        % Процитированные здесь работы:
        %  * подсчитываются, для автоматического составления фразы "Основные результаты ..."
        %  * попадают в авторскую библиографию, при usefootcite==0 и стиле `\insertbiblioauthor` или `\insertbiblioauthorgrouped`
        %  * нумеруются там в зависимости от порядка команд `\printbibliography` в этом разделе.
        %  * при использовании `\insertbiblioauthorgrouped`, порядок команд `\printbibliography` в нём должен быть тем же (см. biblio/biblatex.tex)
        %
        % Невидимый библиографический список для подсчёта количества публикаций:
        \printbibliography[heading=nobibheading, section=1, env=countauthorvak,          keyword=biblioauthorvak]%
        \printbibliography[heading=nobibheading, section=1, env=countauthorwos,          keyword=biblioauthorwos]%
        \printbibliography[heading=nobibheading, section=1, env=countauthorscopus,       keyword=biblioauthorscopus]%
        \printbibliography[heading=nobibheading, section=1, env=countauthorconf,         keyword=biblioauthorconf]%
        \printbibliography[heading=nobibheading, section=1, env=countauthorother,        keyword=biblioauthorother]%
        \printbibliography[heading=nobibheading, section=1, env=countregistered,         keyword=biblioregistered]%
        \printbibliography[heading=nobibheading, section=1, env=countauthorpatent,       keyword=biblioauthorpatent]%
        \printbibliography[heading=nobibheading, section=1, env=countauthorprogram,      keyword=biblioauthorprogram]%
        \printbibliography[heading=nobibheading, section=1, env=countauthor,             keyword=biblioauthor]%
        \printbibliography[heading=nobibheading, section=1, env=countauthorvakscopuswos, filter=vakscopuswos]%
        \printbibliography[heading=nobibheading, section=1, env=countauthorscopuswos,    filter=scopuswos]%
        %
        \nocite{*}%
        %
        {\publications} Основные результаты по теме диссертации изложены в~\arabic{citeauthor}~печатных изданиях,
        \arabic{citeauthorvak} из которых изданы в журналах, рекомендованных ВАК\sloppy%
        \ifnum \value{citeauthorscopuswos}>0%
            , \arabic{citeauthorscopuswos} "--- в~периодических научных журналах, индексируемых Web of~Science и Scopus\sloppy%
        \fi%
        \ifnum \value{citeauthorconf}>0%
            , \arabic{citeauthorconf} "--- в~тезисах докладов.
        \else%
            .
        \fi%
        \ifnum \value{citeregistered}=1%
            \ifnum \value{citeauthorpatent}=1%
                Зарегистрирован \arabic{citeauthorpatent} патент.
            \fi%
            \ifnum \value{citeauthorprogram}=1%
                Зарегистрирована \arabic{citeauthorprogram} программа для ЭВМ.
            \fi%
        \fi%
        \ifnum \value{citeregistered}>1%
            Зарегистрированы\ %
            \ifnum \value{citeauthorpatent}>0%
            \formbytotal{citeauthorpatent}{патент}{}{а}{}\sloppy%
            \ifnum \value{citeauthorprogram}=0 . \else \ и~\fi%
            \fi%
            \ifnum \value{citeauthorprogram}>0%
            \formbytotal{citeauthorprogram}{программ}{а}{ы}{} для ЭВМ.
            \fi%
        \fi%
        % К публикациям, в которых излагаются основные научные результаты диссертации на соискание учёной
        % степени, в рецензируемых изданиях приравниваются патенты на изобретения, патенты (свидетельства) на
        % полезную модель, патенты на промышленный образец, патенты на селекционные достижения, свидетельства
        % на программу для электронных вычислительных машин, базу данных, топологию интегральных микросхем,
        % зарегистрированные в установленном порядке.(в ред. Постановления Правительства РФ от 21.04.2016 N 335)
    \end{refsection}%
    \begin{refsection}[bl-author, bl-registered]
        % Это refsection=2.
        % Процитированные здесь работы:
        %  * попадают в авторскую библиографию, при usefootcite==0 и стиле `\insertbiblioauthorimportant`.
        %  * ни на что не влияют в противном случае
        \nocite{vakbib2}%vak
        \nocite{patbib1}%patent
        \nocite{progbib1}%program
        \nocite{bib1}%other
        \nocite{confbib1}%conf
    \end{refsection}%
        %
        % Всё, что вне этих двух refsection, это refsection=0,
        %  * для диссертации - это нормальные ссылки, попадающие в обычную библиографию
        %  * для автореферата:
        %     * при usefootcite==0, ссылка корректно сработает только для источника из `external.bib`. Для своих работ --- напечатает "[0]" (и даже Warning не вылезет).
        %     * при usefootcite==1, ссылка сработает нормально. В авторской библиографии будут только процитированные в refsection=0 работы.
}

Результаты также направлены в 1 печатное издание в журнал Chaos (IF=3.741, Manuscript ID: CHA22-AR-00995), индексируемый
в Web of Science и Scopus, в котором получены положительные
рецензии, на данный момент вносятся коррективы в соответствии
с комментариями рецензетов. Также направлена заявка на регистрацию РИД
<<Программа для моделирования эволюции вероятности и 
расчета среднего времени поглощения в антагонистической игре двух игроков, управляющих 
случайным блужданием на решетке>>.

% При использовании пакета \verb!biblatex! будут подсчитаны все работы, добавленные
% в файл \verb!biblio/author.bib!. Для правильного подсчёта работ в~различных
% системах цитирования требуется использовать поля:
% \begin{itemize}
%         \item \texttt{authorvak} если публикация индексирована ВАК,
%         \item \texttt{authorscopus} если публикация индексирована Scopus,
%         \item \texttt{authorwos} если публикация индексирована Web of Science,
%         \item \texttt{authorconf} для докладов конференций,
%         \item \texttt{authorpatent} для патентов,
%         \item \texttt{authorprogram} для зарегистрированных программ для ЭВМ,
%         \item \texttt{authorother} для других публикаций.
% \end{itemize}
% Для подсчёта используются счётчики:
% \begin{itemize}
%         \item \texttt{citeauthorvak} для работ, индексируемых ВАК,
%         \item \texttt{citeauthorscopus} для работ, индексируемых Scopus,
%         \item \texttt{citeauthorwos} для работ, индексируемых Web of Science,
%         \item \texttt{citeauthorvakscopuswos} для работ, индексируемых одной из трёх баз,
%         \item \texttt{citeauthorscopuswos} для работ, индексируемых Scopus или Web of~Science,
%         \item \texttt{citeauthorconf} для докладов на конференциях,
%         \item \texttt{citeauthorother} для остальных работ,
%         \item \texttt{citeauthorpatent} для патентов,
%         \item \texttt{citeauthorprogram} для зарегистрированных программ для ЭВМ,
%         \item \texttt{citeauthor} для суммарного количества работ.
% \end{itemize}
% % Счётчик \texttt{citeexternal} используется для подсчёта процитированных публикаций;
% % \texttt{citeregistered} "--- для подсчёта суммарного количества патентов и программ для ЭВМ.

% Для добавления в список публикаций автора работ, которые не были процитированы в
% автореферате, требуется их~перечислить с использованием команды \verb!\nocite! в
% \verb!Synopsis/content.tex!. % Характеристика работы по структуре во введении и в автореферате не отличается (ГОСТ Р 7.0.11, пункты 5.3.1 и 9.2.1), потому её загружаем из одного и того же внешнего файла, предварительно задав форму выделения некоторым параметрам

\textbf{Объем и структура работы.} Диссертация состоит из~введения,
\formbytotal{totalchapter}{глав}{ы}{}{} и
заключения. 
% \formbytotal{totalappendix}{приложен}{ия}{ий}{}.
%% на случай ошибок оставляю исходный кусок на месте, закомментированным
%Полный объём диссертации составляет  \ref*{TotPages}~страницу
%с~\totalfigures{}~рисунками и~\totaltables{}~таблицами. Список литературы
%содержит \total{citenum}~наименований.
%
Полный объем диссертации составляет
\formbytotal{TotPages}{страниц}{у}{ы}{}, включая
\formbytotal{totalcount@figure}{рисун}{ок}{ка}{ков} и
\formbytotal{totalcount@table}{таблиц}{у}{ы}{}.
Список литературы содержит
\formbytotal{citenum}{наименован}{ие}{ия}{ий}.
