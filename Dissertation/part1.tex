\chapter{Аналитический обзор процессов случайных блужданий в качестве модели некоторых природных явлений}\label{ch:ch1}

Первая глава посвящена обзору работ в областях хемотаксиса бактерий, случайных блужданий и игровых случайных блужданий. В разделе \cref{sec:ch1/sec1} проводится описание способов движения живых организмов и механизм изменения подвижности в зависимости от внешней концентрации веществ. В разделе \cref{sec:ch1/sec2} приводится история развития области случайных блужданий и демонстрируется широкий спектр приложений в различных областях науки. В заключительном разделе \cref{sec:ch1/sec3} рассматриваются вопросы возникновения случайных блужданий в рамках игровых механик, и описываются возможности применения мобильных технологий для проведения полевых экспериментов.

\section{Таксис и случайные блуждания}\label{sec:ch1/sec1}

\subsection{Подвижность живых организмов}\label{subsec:ch1/sec1/sub1}

Подвижность -- это способность клетки или организма перемещаться в пространстве за счет затрат энергии. Живые организмы используют различные способы для осуществления движений, такие как движение с помощью ресничек и жгутиков, амебоидное движение, мышечное сокращение, скользящая подвижность и роевая подвижность. Большинство животных подвижны и используют для перемещения такие способы, как ходьба, скольжение, плавание и полет. Движение наблюдается на различных пространственных и временных масштабах, в различных средах, с использованием различных механизмов, при большом разнообразии размеров и форм организмов. 

Подвижность является важным фактором выживания видов, с помощью которого живые организмы адаптируются к условиям окружающей среды, получают питательные вещества, спасаются от токсинов и хищников и обмениваются генетической информацией посредством спаривания. Геномные исследования демонстрируют наличие 18 типов молекулярных механизмов движения \cite{miyata_tree_2020}. Развитие технологий позволило отслеживать траектории движения организмов в их естественной среде обитания в течение длительного времени \cite{boyer_modelling_2010}. Движение организмов в большинстве случаев происходит ввиду необходимости взаимодействия с окружающей средой, изменения которой достаточно трудно предсказуемы. Однако, вопросы исследования движения организмов обычно возникают именно в контексте решения живым организмом конкретной задачи: поиск пищи, движение в сторону дома, поиск пары для спаривания или объединение в колонии \cite{zaburdaev_levy_2015}. 

Для описания движения организмов исследователи активно применяют концепцию случайных блужданий. Наиболее успешным при описании поведения движения различных простых организмов, таких как бактерии и насекомые, оказался способ описания, основанный на использовании в решении только локальной информации в пространстве и времени, доступной организму \cite{turchin_quantitative_1998}. Учет распределений с тяжелым хвостом для длительностей направленного движения позволил описать движение микрозоопланктона, паукообразных обезьян, морских хищников, полета альбатросов и множества других видов \cite{boyer_modelling_2010} с использованием модели случайных блужданий Леви. Исследования движения показали, что полеты Леви, чередующиеся с броуновским движением, описывают процесс охоты животных \cite{sims_scaling_2008}: отсутствие возможности найти пищу приводит к переключению режима с броуновского движения на полеты Леви. Развитие теории случайных блужданий позволило дополнительно учесть и конечную скорость движения агентов, что привело к созданию модели случайных блужданий Леви \cite{shlesinger_random_1982}.   

Однако, ограничения марковского подхода заставляют исследователей искать новые способы описания процессов, что приводит к возникновению новых моделей движения, обобщающих различные организмы и их свойства \cite{nathan_movement_2008}. Подход к построению динамических моделей для описания движения различных конкретных видов организмов активно применяется в различных областях биологии \cite{berg_random_1993}. Сложности с использованием марковского подхода возникают при попытке анализа движения видов, обладающих способностью запоминать информацию и использовать ее для предсказания будущего. Использование эпизодической памяти, когнитивных карт местности, способности к оценке стоимости движения приводит к тому, что движение организма не обязательно полностью управляется случайной компонентой, основанной на ближайших состояниях, но также может основываться на анализе затрат и выгод для различных вариантов движения \cite{boyer_modelling_2010}.

\subsection{Стратегия движения <<run-and-tumble>>}\label{subsec:ch1/sec1/sub2}

Стратегия движения <<run-and-tumble>> наблюдается у некоторых видов бактерий и других микроскопических агентов \cite{berg_coli_2004}. Стратегия состоит из чередующихся этапов <<run>> и <<tumble>>, соответствующих движению в фиксированном (или медленно меняющемся) направлении, и вращению на одном месте. Второй этап состоит в переориентации агента и выборе нового направления для последующего движения. При этом вращение происходит случайным образом в соответствии с некоторой плотностью вероятности направлений, зависящей от локальной среды и состояния организма. Длительность движения в фиксированном направлении аналогично является случайным процессом. 

Наиболее известным примером агента, использующего стратегию <<run-and-tumble>>, является бактерия E.~coli \cite{berg_chemotaxis_1972}. Для плавания E.~coli использует несколько вращающихся жгутиков в одном из двух направлений: по часовой стрелке или против часовой стрелки. Одновременное вращение всех жгутиков против часовой стрелки приводит бактерию к движению по примерно прямой траектории (этап <<run>>). Изменение направления вращения одного или несколько жгутиков на вращение по часовой стрелке, приводит к изменению повороту бактерии (этап <<tumble>>) \cite{turner_realtime_2000}. Плотность распределения угла поворота для E.~coli имеет среднее значение около $62^\circ$ \cite{berg_chemotaxis_1972}. Многие виды бактерий, особенно имеющие один жгутик, полностью меняют направление своего движения после переключения вращения жгутика, что приводит к паттерну движения <<run-reverse>> \cite{theves_bacterial_2013}.

\subsection{Хемотаксис бактерий}\label{subsec:ch1/sec1/sub3}

Адаптация бактерий к различным условиям окружающей среды возможна благодаря разнообразию размеров, форм, способов передвижения и способности к высокой чувствительности концентрации веществ. Отслеживание бактерией концентрации питательных веществ за счет механизма хемотаксиса приводит к движению в направлении аттрактанта, или в направлении от различных вредоносных веществ -- репеллентов. Различение веществ бактерией осуществляется посредством связывания молекул вещества с рецептивными участками специфичных хеморецепторов. Например, бактерия E.~coli обладает галактозочувстительными и рибозочувствительными хеморецепторами \cite{vorotnikov_chemotaxis_2011}.

Белковые механизмы внутри цитоплазмы бактерии передают сигнал от рецепторов к двигательному аппарату, обеспечивающему движение клетки в определенном направлении. Управление движением клетки осуществляется за счет вращения жгутиков по часовой стрелки либо против часовой стрелки. Переключение направления вращения жгутиков является ключевым механизмом управления хемотаксической активностью E.~coli и многих других бактерий. Для ориентации при перемещении в среде, бактерия проводит измерение изменений химической концентрации во времени. Характерная скорость движения E.~coli составляет 10--20 своих длин в секунду \cite{milo_cell_2015}. Механизм сравнения текущей концентрации с измерением, полученным несколько секунд назад, позволяет клетке оценить разницу концентраций определенного вещества на расстоянии, во много раз превышающем длину самой клетки. Реализация такого механизма возможна благодаря адаптивному метилированию хеморецепторов, зависящем от количества связанных лигандов \cite{strayer_biohimiya_1984}. Возникающая задержка от момента связывания молекулы вещества до момента метилирования выполняет функцию памяти.

Хотя паттерн движения бактерии остается неизменным, регуляция движения осуществляется за счет изменения длительностей движения бактерии на этапе <<run>>. Движение в сторону роста концентрации привлекающих веществ или в направлении убывания концентрации отталкивающих веществ приводит к увеличению длительности этапа <<run>> \cite{berg_chemotaxis_1972}. На движение каждой клетки влияет броуновское движение, что приводит к отклонению от выбранного направления движения. Однако такой механизм позволяет колонии бактерий в среднем обладать направлением движения в сторону аттрактанта или от репеллента. Для описания этих процессов была предложена линейная теория хемотаксиса де Женна \cite{de_gennes_chemotaxis_2004}. Условия применимости теории ограничены малым постоянным градиентом концентрации химического вещества.

\section{Случайные блуждания и полеты Леви}\label{sec:ch1/sec2}

Первое появление термина <<случайные блуждания>> относится к письму Карла Пирсона в редакцию журнала Nature в 1905 году \cite{pearson_problem_1905}. Мотивированный проблемами, возникшими в биологии, он сформулировал следующую задачу: <<Человек начинает путешествие из точки $O$ и проходит $l$ ярдов по прямой линии. Затем он поворачивается на какой угодно угол, и проходит еще $l$ ярдов по прямой линии. Он повторяет этот процесс $n$ раз. Я ищу вероятность того, что после этих $n$ прогулок он окажется на расстоянии между $r$ и $r+dr$ от своей начальной точки $O$>>.

На письмо ответил британский физик Лорд Рэлей \cite{rayleigh_problem_1905}, который ранее в своей работы 1880 года от теории звука рассмотрел $n$ изопериодических колебаний единичной амплитуды и фазы, распределенные случайным образом \cite{rayleigh_lxi_1877}. Для достаточно больших $n$ Рэлей нашел асимптотическое решение в замкнутой форме, отражающее распределение вероятностей в задаче Карла Пирсона. 

Решение задачи для многомерного случая в евклидовом пространстве представляется существенно более сложной задачей в связи с чем в своих работах Лорд Рэлей рассматривал модель случайного блуждания на решетке. Исследуя случайные блуждания на решетке Дьердь Пойа в своих работах 1919 и 1921 года \cite{polya_uber_1921} показал, что, в случае случайного блуждания на решетке с равновероятными переходами во всех направлениях в одномерном и двумерном случаях, агент возвращается в исходную точку с вероятностью $1$, а для размерностей больше двух с вероятностью $0$.

Как заметил профессор прикладной математики Невилл Темперлей, различные комбинаторные задачи, возникающие при изучении разнообразных физических явлений, связанных с решетками, могут быть разрешимы в терминах подсчета числа траекторий случайного блуждания с ограничениями \cite{temperley_combinatorial_1956}. Задачи, возникающие в различных областях таких как химия, физика, экономика, биология, сводились к анализу случайных блужданий на решетках со сложной структурой и особыми ограничениями \cite{kuhn_uber_1930, flory_principles_1953, gee_interaction_1946, ising_beitrag_1925, tricomi_funzioni_1954}. 

После публикации Куна в начале 20-го века все большее внимание уделялось статистике конфигураций гибких макромолекул \cite{kuhn_uber_1930}. Важность исследования состояла в том, что химические свойства и биологические функции различных макромолекул напрямую зависят от их трехмерной пространственной конфигурации. Кун рассматривал первое приближение такой задачи с помощью свободно сочлененной цепи со звеньями фиксированной длины, но со случайной ориентацией, то есть задача Пирсона, но в трехмерном случае. Такая конструкция не учитывает тот факт, что в одной точке пространства может быть только один атом полимерной цепи. Упрощение задачи на случай решеток, подобно тому, как это делал Лорд Рэлей, позволяет получить решение с помощью численных методов.

Первые попытки в направлении исследования блужданий без самопересечений были сделаны В. Орром в 1946 году \cite{gee_interaction_1946}. В 1924 году немецкий и американский математик Эрнст Изинг сформулировал модель ферромагнетизма на плоскости, для которой интересным является ее соответствие случайным блужданиям без самопересечений на решетке \cite{ising_beitrag_1925}. В 1941 году Бартель Леендерт ван дер Варден показал, как можно свести эту задачу к подсчету числа решетчатых графов, состоящих из замкнутых многоугольников \cite{van_der_waerden_lange_1941}. Знаменитое решение задачи Изинга, полученное Ларсом Онзагером, было впервые опубликовано 18 февраля 1942 года в дискуссионных заметках на собрании Нью-Йоркской академии наук. Модель Изинга может быть использована для описания и других различных физических систем, таких как поглощение газа поверхностью или описание двухкомпонентного сплава. Эти явления описываются грубой моделью равновесия жидкости и пара на двумерной решетке.

В 1912 году Андрей Андреевич Марков разработал общую постановку задачи случайных полетов, а также предложил метод для отыскания решения \cite{markov_wahrscheinlichkeitsrechnung_1912}. Задача формулируется так: есть частица в трехмерном пространстве, которая совершает перемещения на каждом шаге в соответствии с некоторым распределением, зависящим от номера шага. Требуется найти вероятность обнаружить частицу на некотором расстоянии от позиции старта. Решение Маркова основано на применении свойств преобразования Фурье.

Исследуя свойства случайных процессов в своей работе 1907 года А.~А.~Марков выделил класс обладающий характеристикой независимости будущих состояний от прошлых состояний при определенном настоящем состоянии \cite{shiryaev_2021}. В зависимости от рассматриваемой модели Марковские процессы могут быть обобщены на процессы с дискретным и непрерывным временем. Марковские цепи высшего порядка в свою очередь обладают свойством зависимости перехода от последних $k$ состояний.

Анализ различных типов случайных блужданий позволил выделить множество различных объектов, на которых возможно осуществлять блуждание: графы, числовая прямая (целые или действительные числа), векторные пространства, конечные группы, группы Ли, кривые поверхности и другие. Случайные блуждания также различаются по типу времени: дискретное или непрерывное $(t \in [0; +\infty])$. Хотя в общем случае случайное блуждание необязательно должно обладать свойством марковости, по умолчанию подразумевают зависимость будущих состояний только от текущего состояния.

Теория случайных блужданий развивалась независимо в нескольких областях исследований: биология \cite{chowdhury_100_2005}, теория вероятности \cite{strecker_alexandr_2011}, финансы \cite{bachelier_theorie_1900} и физика \cite{pearson_problem_1905,rayleigh_problem_1905}. Такое разнообразие приложений показывает, что данный процесс лежит в основе многих фундаментальных знаний о нашем мире.

\subsection{Нормальная диффузия и броуновское движение}\label{subsec:ch1/sec2/sub1}

Шотландский ботаник Роберт Браун в 1827 г. наблюдал под микроскопом помещенные в воду крошечные крупинки цветочной пыльцы \cite{brown_brief_2015}. Он заметил, что эти крупинки совершают крайне беспорядочные, зигзагообразные движения. Согласно его же словам, наблюдаемые движения <<не связаны с потоками в жидкости, с испарением, а присущи самим частицам>>. Впоследствии это движение было названо броуновским движением. Суть же этого движения была понята позднее. Объяснение состояло в том, что молекулы жидкости производят огромное число хаотических ударов по частицам за сколь угодно малый интервал времени.

В 1831-1833 годах Томас Грэм впервые провел систематическое экспериментальное исследование диффузии в газах. На основе измерений Грэма в 1855 году Адольф Фик предложил формализм для описания закона диффузии. Математическое уравнение диффузии было известно с 1822 года, полученное французским математиком Жан-Батистом Жозефом Фурье для описания явления теплопроводности \cite{fourier_theorie_1822}. 

Работы Эйнштейна и Смолуховского положили начало строгого и количественного описания, связывающего микроскопическую и макроскопическую динамику частиц. Механизм броуновского движения был впервые объяснен в 1905 году Альбертом Эйнштейном \cite{einstein_uber_1905}. Близкими к работе Эйнштейна были работы Мариана Смолуховского 1906 года \cite{vonsmoluchowski_zur_1906}. Идея метода Эйнштейна состояла в том, что для математического изучения броуновского движения он применяет вероятностно-статистический подход, показывая, что плотность вероятности положения частицы по каждой из координат в момент времени $t$ удовлетворяет уравнению теплопроводности, из чего он заключает, что за время $t$ частица смещается на расстояние порядка $\sqrt{t}$. Во второй части работы Эйнштейн показывает зависимость коэффициента диффузии $D$ от ряда физических величин (числа Авогадро, температуры и вязкости жидкости).

Работа Эйнштейна давала косвенное доказательство существования атомов и молекул. В 1908 году французский физик Ж. Перрен экспериментально подтвердил в своих работах существование атомов \cite{perrin_atomes_1921}. Объяснение свойств траекторий броуновского движения частиц было продолжено американским математиком Н. Винером в 1920 – 1930-х годах, в честь которого броуновское движение называют также винеровским процессом \cite{wiener_average_1921a}.

\subsection{Аномальная диффузия и случайные блуждания Леви}\label{subsec:ch1/sec2/sub2}

С точки зрения понимания движения физических объектов уравнение диффузии имело один недостаток: по прошествии бесконечно малого промежутка времени частица имеет ненулевую вероятность оказаться на любом расстоянии от места старта. То есть в соответствии с уравнением диффузии частицы могут иметь бесконечную скорость распространения.

Для решения этой проблемы в 1922 году английский физик Джефри Играм Тейлор построил модель случайного блуждания для описания задачи турбулентной диффузии в гидродинамике \cite{taylor_diffusion_1922}. Он рассмотрел случай, в котором движение частицы имеет конечную скорость на прямолинейных участках зигзагообразной траектории. Аналогичную проблему решал Райнхольд Фюрт в 1920 году в приложении к описанию постоянного броуновского движения \cite{furth_zur_1920}. Оба подхода предполагали, что перемещение частицы ограничено конусом ее максимальной скорости.

Описание такого движения дифференциальными уравнениями было представлено Давыдовым Б. И. в 1934 году в докладе академии наук СССР \cite{davydov_diffusion_1934}. Телеграфное уравнение позволило ему ввести ограничения на скорость распространения частиц. Впервые уравнения такого типа были получены ранее в работах немецкого физика Густава Роберта Кирхгоффа и английского физика Оливера Хэвисайда при исследовании передачи электрического тока по линиям электропередач \cite{kirchhoff_ueber_1857}.

В 1951 году в своей работе британский математик Сидней Гольдштейн вывел из модели случайного блуждания, предложенной Тейлором, телеграфные уравнения \cite{goldstein_diffusion_1951}. Уравнения Тейлора-Гольдштейна используются в области геофизической гидродинамики, основной целью которой является разработка численных методов для составления прогноза погоды и опасных природных явлений, а также изменений в геомагнитном поле.

Американские математики Эллиот Уотерс Монролл и Джордж Герберт Вайс в 1965 году разработали новую модель случайного блуждания с непрерывным временем \cite{montroll_random_1965}. В их описании движения кроме самого перемещения появилась ещё одна стадия -- случайная задержка между прямолинейными участками. Такая модель позволила обосновать аномальную диффузию, при которой частицы распространялись по пространству медленнее, чем при броуновском движении (субдиффузия).

Существование аномальной диффузии в турбулентных потоках было предложено в 1926 году английским математиком Льюисом Фраем Ричардсоном \cite{richardson_atmospheric_1997}. Используя метеозонды, он продемонстрировал, что в атмосфере возникает турбулентная супер-диффузия и частицы распространяются быстрее, чем в случае нормальной диффузии.

В своей работе советские математики Борис Владимирович Гнеденко и Андрей Николаевич Колмогоров в 1949 году построили модель, позволяющую объяснить это явление \cite{gnedenko_predelnye_1949}. Чтобы получить модель случайного блуждания с супер-диффузионным транспортом необходимо разрешить очень длинные прямолинейные отрезки, чтобы частица могла далеко уходить от своего местоположения. Обычную диффузию ограничивает центральная предельная теорема, которая гласит, что сумма случайных величин (отрезков пути блуждающего агента), имеющих конечный второй момент, распределена по нормальному закону. Поэтому необходимо выбрать распределение для длин участков пути, имеющее бесконечную дисперсию. Таким распределением является, например, степенное распределение с тяжелым хвостом. Математики установили, что устойчивые распределения, полученные французским математиком Полем Пьером Леви в 1937 году, и супер-диффузия взаимосвязаны \cite{levy_theorie_1937}. В своей книге <<Фрактальная геометрия природы>> французский математик Бенуа Мандельброт в 1982 году дал название модели случайного блуждания с длительными направленными отрезками движения: <<полеты Леви>> \cite{mandelbrot_fractal_1983}. Паттерн случайного блуждания с длинными перелетами, сменяющимися короткими, в модели полетов Леви наблюдается на любом масштабе рассмотрения траектории. Такое свойство характеризует фрактальную безмасштабную структуру траектории движения.

В описании схемы полетов Леви присутствует понятие полета или прыжка, когда частица за нулевой промежуток времени сразу оказывается в следующей точке пространства, но так, что средний квадрат длины прыжка бесконечен. Поэтому полеты Леви, как и описание броуновского движения, обладают одной и той же проблемой -- распространение частицы с бесконечной скоростью. Сложность сопоставления модели и экспериментальных данных не позволяла подобрать параметры модели для точного описания. Поэтому в дальнейших работах была разработана модель случайных блужданий Леви, в которой скорость распространения ограничивается некоторым максимальным значением. 

Американский физик Майкл Ф. Шлезингер и израильский физик Джозеф Клафтер в 1986 году одними из первых предположили, что в моделях движения живых существ могут наблюдаться блуждания Леви \cite{shlesinger_levy_1986}. Они показали, что фрактальные свойства могут быть полезны при поиске (пищи, убежища, лучших условий среды), так как исключают ненужное повторное посещение уже пройденной территории, а соответственно организм становится более приспособленным с точки зрения естественного отбора. Удивительная идея ученых предполагала, что блуждания Леви могут быть врожденной и закрепленной в процессе эволюции оптимальной стратегией поиска.

Идея получила распространение после статьи Гандхимохана Вишванатана 1996 года \cite{viswanathan_levy_1996}. В ней авторы описали траектории полета альбатроса Diomedea exulans (странствующий альбатрос) при поиске пищи на поверхности океана в их естественной среде обитания. Ученые показали инвариантность траекторий относительно масштаба рассмотрения, а также что распределения длительностей перелетов распределены по степенному закону. 

В следующей своей работе 1999 года Г. Вишванатан математически показал, что агенты, использующие для поиска пищи паттерн блужданий Леви, действительно могут быть более успешны, чем если бы они использовали другие паттерны движения \cite{viswanathan_optimizing_1999}. Наиболее явным образом эффект от блужданий Леви проявляется при динамической среде, в которой распределение пищи по пространству меняется во времени. Несмотря на рост интереса к предложенным идеям поиска пищи, в работах Вишванатана были обнаружены неточности в связи с использованием неподходящих статистических методов, которые в дальнейшем были исправлены. Дальнейший сбор сведений о разнообразных живых организмах показал, что многие из них имеют в своей основе движения механизм блужданий Леви \cite{zaburdaev_levy_2015}.

Теория случайных блужданий имеет широкий спектр применений в разных областях исследований. Так, например, в 2018 году математики показали, что процесс появления новых научных и инновационных идей также можно описать моделью случайного блуждания с усилением \cite{iacopini_network_2018}.

\section{Антагонистические игры и случайные блуждания}\label{sec:ch1/sec3}

Теория игр тесно переплетается с проблемами случайных блужданий в широком спектре областей прикладных и фундаментальных исследований от хемотаксиса бактерий \cite{zaburdaev_levy_2015,bib1,bib2} и рыночных взаимодействий \cite{li_evolutionary_2013,friedman_towards_2001} до блужданий роя автономных роботов \cite{marques_particle_2006,xiong_intelligent_2008} и компьютерных игр \cite{outlaw_markov_2016,dankel_long_2004,dshalalow_random_2008}. Игры, в которых возникают случай влияет на развитие игровой ситуации, могут быть рассмотрены как случайное блуждание на некотором графе всех возможных конфигураций предметов игры или в некотором непрерывном пространстве всех возможных состояний. В общем случае окончание таких игр определяется при возникновении некоторой предопределенной выигрышной траектории. Одной из таких возможных ситуаций является поглощение траектории в некотором состоянии при случайном блуждании в пространстве конфигураций. В зависимости от типа игры игрокам может требоваться оптимизировать длительность случайного блуждания или расставить поглощающие состояния так, чтобы раньше другого игрока поймать траекторию \cite{baldi_intransitiveness_2020}. Одной из самых простых игр такого типа является задача о разорении игрока \cite{feller_introduction_1968} и различные ее модификации \cite{baldi_intransitiveness_2020,cencetti_second_2016,kittas_trapping_2008,lee_random-walk_1989}. Обобщение случайных блужданий игрового типа было проведено в работе И.~В.~Романовского \cite{romanovsky_1961}. Рассматриваемые теорией игр процессы обусловлены взаимодействием одного игрока с некоторой системой, двух игроков или множества игроков. В последующих разделах проведен анализ первых двух случаев, соответствующих антагонистическим играм между двумя игроками с противоположными интересами.

\subsection{Задача о разорении игрока}\label{subsec:ch1/sec3/sub1}

Впервые упоминание о задаче разорения игрока появилось в переписке Блеза Паскаля и Пьера Ферма в 1656 году при рассмотрении игры тремя костями между двумя игроками \cite{renyi_1980}. Первый игрок получал очки при выпадении суммы на игральных костях равной $11$, а второй игрок при выпадении суммы равной $14$. Однако, способ начисления очков был модифицирован: очко добавляется к счету игрока только в том случае, если счет его противника равен нулю, а в противном случае очко будет вычтено из счета его противника. В этом случае счет отстающего игрока всегда остается равным нулю. Первый из игроков набравший $12$ очков объявлялся победителем. В переформулированной версии письма Пьером де Каркави, направленной Христиану Гюйгенсу в 1656 году, вопрос к задаче состоял в нахождении вероятности победы первого и второго игрока. 

В своей работе <<De Ratiociniis in Ludo Aleae>> Христиан Гюйгенс \cite{hald_history_2003,huygens_christiani_1714} получил более приближенную формулировку к классической: игроки начинают с $12$ очков, успешный бросок  трех костей ($11$ для первого и $14$ для второго) добавляет этому игроку одно очко и вычитает у второго, при этом первый достигший нуля очков проигрывает. Вопрос, как и ранее, состоял в нахождении вероятности победы игроков. 

Классическая формулировка обобщает для одномерного случая задачу о разорении игрока для случая произвольных начальных условий и произвольных вероятностей перехода \cite{feller_introduction_1968}. Пусть у первого игрока есть $-A$ монет ($A < 0, -A > 0$), у второго игрока -- $B$ монет. На каждом ходу подбрасывается ассиметричная монета, имеющая вероятность выпадения аверса $p$ и реверса $1-p$. При выпадении аверса одна монета переходит от второго игрока к первому, при выпадении реверса -- наоборот. Требуется найти вероятность проигрыша за $n$ шагов, а также общую вероятность проигрыша каждого из игроков. В дополнение к формулировке Христиана Гюйгенса ставится вопрос о среднем времени игры, где время игры характеризует количество ходов до проигрыша одного из игроков.

Тесная связь со случайными блужданиями позволяет сформулировать данный процесс в виде дискретного блуждания частицы по целочисленному отрезку $[A; B]$, при этом выход на границу отрезка характеризует проигрыш одного из игроков. \cite{shiryaev_2021} 

Решение задачи состоит в рассмотрении схемы Бернулли с последовательностью бернуллиевских случайных величин $\xi_i$ с вероятностью $p$ дающих $+1$ и с вероятностью $q=1-p$ значение $-1$. Тогда сумма таких величин $S_k=\sum_{i=1}^{k} \xi_i$ равна случайной величине, соответствующей положению частицы в случайном блуждании на отрезке $[A; B]$. Вводя обозначения для вероятностей завершить игру в точках $A$ и $B$ за время $[0; k]$ при старте в позиции $x$ соответственно $\alpha_k(x), \beta_k(x)$ запишем рекуррентные соотношения:
\begin{equation}
    \label{eq:eq1}
    \begin{alignedat}{2}
        \alpha_k(x) = p\alpha_{k-1}(x+1)+q\alpha_{k-1}(x-1),\\
        \beta_k(x) = p\beta_{k-1}(x+1)+q\beta_{k-1}(x-1).
    \end{alignedat}
\end{equation}

При достаточно больших $n$ решение рекуррентного соотношения близко к стационарной точке точечного отображения при заданных граничных условиях:
\begin{equation}
    \label{eq:eq2}
    \begin{alignedat}{2}
        \alpha(x) = p\alpha(x+1)+q\alpha(x-1), \alpha(A)=1, \alpha(B)=0,\\
        \beta(x) = p\beta(x+1)+q\beta(x-1), \beta(A)=0, \beta(B)=1.
    \end{alignedat}
\end{equation}

Поиск решения уравнения в форме $\frac{q}{p}^{x}$ дает следующий результат:
\begin{equation}
    \label{eq:eq3}
    \begin{alignedat}{2}
        \alpha(x) = (\frac{q}{p}^B-\frac{q}{p}^x)/(\frac{q}{p}^B-\frac{q}{p}^A),\\
        \beta(x) = (\frac{q}{p}^x-\frac{q}{p}^A)/(\frac{q}{p}^B-\frac{q}{p}^A).
    \end{alignedat}
\end{equation}

При справедливой игре $(p=q)$ выражения для определения вероятностей редуцируются до линейной формы:
\begin{equation}
    \label{eq:eq4}
    \begin{alignedat}{2}
        \alpha(x) = (B-x)/(B-A),\\
        \beta(x) = (x-A)/(B-A),
    \end{alignedat}
\end{equation}
где $x \in [A; B]$ -- стартовое положение на отрезке.

Заключительный аспект задачи о разорении игрока состоит в исследовании среднего времени достижения финального состояния. С точки зрения случайных блужданий процесс представляется в виде Марковской цепи с поглощающими состояниями на концах отрезка и промежуточными состояниями в целочисленных координатах отрезка. Рассмотрим подход к решению задачи о нахождении математического ожидания времени окончания игры $m_k(x)$ для некоторой игры длины $k$ находящейся в состоянии $x$. Тогда рекуррентное соотношение для соседних состояний представляется в виде:
\begin{equation}
    \label{eq:eq5}
    m_k(x) = p m_{k-1}(x + 1) + q m_{k-1}(x - 1) + 1, x \in (A, B), k > 0,
\end{equation}

На границе в точках $A$, $B$ количество ходов для завершения игры равно нулю, то есть $m_k(A) = m_k(B) = 0$. С учетом конечности математического ожидания пределом при $k \xrightarrow \infty$ будет являться решение рекуррентного соотношения $m(x)=p m(x+1) +q m(x - 1) + 1$. Итоговое решение для общего случая представляется в виде:
\begin{equation}
    \label{eq:eq6}
    m(x) = \frac{1}{p - q} (B \beta(x) + A \alpha(x) - x)
\end{equation}

При симметричной монетке формула упрощается до $m(x) = (B - x) (x - A)$, а в случае равного начального капитала $m(x) = B^2$.

Таким образом оценки позволяют найти как вероятность выигрыша каждого из игроков в зависимости от параметров игры, так и среднее время игры.

Обобщение задачи о разорении игрока на двумерный случай было рассмотрено израильскими математиками в 1994 году \cite{orr_computer_1994}. Для решения задачи был применен метод производящих функций и получены выражения для случая с равновероятными переходами на отрезке и на квадратной решетке. Применяя символьные вычисления в пакете MAPLE \cite{monagan2012maple} тем же способом было найдено решение с произвольными вероятностями перехода в 4 направлениях на четырех-связной решетке. 

Естественным продолжением двумерного случая является обобщение на произвольные размерности. В 2000 году Марко Петковсек и Андрей Кмет рассмотрели задачу о разорении игрока, в которой имеется несколько различных валют \cite{kmet_gamblers_2002}. Формулировка игровой механики для многомерного случая состоит в равновероятном выборе валюты и победителя на каждом ходу. В результате хода победитель забирает одну монету выпавшей валюты у проигравшего. Игра продолжается до тех пор, пока у одного из игроков не закончатся монеты любой из валют. В своей работе словенские математики получили среднее время игры в явной форме для случая равновероятных переходов между соседними узлами решетки с применением дискретного преобразования Фурье и выражений для спектра симметричной трехдиагональной матрицы Теплица. Полученное выражение для среднего времени игры в зависимости от начального капитала игроков состоит из двойной суммы для двумерного случая и может быть вычислено за время $O(N^2)$: 
\begin{equation}
    \begin{aligned}
    a_{ij} =& \frac{4}{N^2} \sum_{\substack{k=1\\k\,\textup{-неч.}}}^{N - 1} \sin \left (\frac{jk\pi}{N} \right )\cot \left (\frac{k\pi}{2N}\right ) \\
    &\sum_{\substack{l=1\\l\,\textup{-неч.}}}^{N - 1} \sin \left (\frac{il\pi}{N} \right )\cot \left (\frac{l\pi}{2N} \right ) \Bigg / \left (\sin^2 \left (\frac{k\pi}{2N} \right )+\sin^2 \left (\frac{l\pi}{2N} \right ) \right ), \\
    &0 \leq i, j \leq N
    \end{aligned}
    \label{eq:eq7}
\end{equation}
    
Хотя авторы не получили более быстрый алгоритм для общего случая, высказано предположение о возможности получения выражения в виде одной суммы. Явные формулы в виде одной суммы (вычислительная сложность $O(N)$) были получены для частного случая стартовых капиталов одинакового размера являющегося степенью 2. Также математики получили решение для многомерного случая на базе уравнения Сильвестра и тензорного спектрального разложения.

\subsection{Случайные блуждания игрового типа}\label{subsec:ch1/sec3/sub2}

Дальнейшие исследования игр привели к анализу многошаговых процессов, в которых игроки на каждом ходу выбирают одну из доступных стратегий \cite{bellman_decision_1954}. Ричард Беллман в своей работе 1954 года рассматривал проблему принятия решений в многошаговых играх в условиях неопределенности \cite{bellman_decision-making_1954} и привел приближенное решение задачи. Возникновение прямого конфликта между игроками позволяет использовать результаты известных работ по теории игр с нулевой суммой \cite{nash_non-cooperative_1951}, тогда как частичное противостояние требует более сложного анализа игр с ненулевой суммой \cite{bellman_non-zero_1949}. 

Одной из естественно возникающих вариаций являются игры на выживание, введенные в работах \cite{hausner_games_1952,peisakoff_more_1952} в 1952 году Мелвином Хауснером и Мелвином Пейсахов. В таких многошаговых схемах игроки обладают ограниченным ресурсом, по иссякании которого происходит завершение игры. В зависимости от количества начального числа монет оценивалась вероятность победы одного из игроков при некоторых фиксированных смешанных стратегиях игроков, определяемых до начала игры. Беллман показал, что в игре с нулевой суммой при достаточно большом стартовом капитале с большим числом ходов оптимальная стратегия игрока приблизительно такая же, как и в случае одношаговой игры, в которой оба игрока максимизируют математическое ожидание выигрыша. 

Несмотря на полученный успех в нахождении приближенного решения многошаговой игры, требовался дальнейший анализ оптимальных стратегий. Используя теорию семимартингалов американские математики Ллойд Шепли и Джон Милнор в своей работе <<Об играх на выживание>> в 1957 году \cite{milnor_games_1956} продемонстрировали решение функционального уравнения многошаговой игры на выживание. 

Игровая динамика многошаговых схем предполагает возможность принятия решения игроками на каждом ходу, однако в варианте Беллмана имеет строго детерминированный исход при сделанных выборах игроков. И.~В.~Романовский указал на возможность обобщения игры с учетом влияния случайной компоненты на выигрыш игроков в каждом ходе \cite{romanovsky_1961}. В работе рассмотрены различные варианты игры: игра с постоянной суммой, игра с бесконечным капиталом одного из игроков, и многомерные блуждания. Для различных типов игр были предложены решения функционального уравнения, как для вероятности победы игрока, так и для среднего времени игры.

\subsection{Мобильные приложения в полевых экспериментах}\label{subsec:ch1/sec3/sub3}

Аналитические работы математиков по теории игр построили фундаментальные основы для игровых процессов и позволили найти решение к некоторым классам игр. Развитие вычислительной техники и рост производительности позволил решать задачи не только на бумаге, проводя сложный анализ функциональных уравнений, но и применяя принципы численного моделирования и симуляции Монте-Карло \cite{Sobol_monte-carlo_1968} для решения задач фиксированной размерности с конкретными значениями параметров. Появление интернета в 1983 году создало фундамент для построения взаимодействия между большим количеством людей независимо от их географического расположения \cite{Barbruk_internet_2015}. Внедрение цифровой мобильной связи GSM, ее эволюция и широкое внедрение по всему миру привело к возникновению технологий взаимодействия между пользователями посредством текстовых сообщений, аудио сообщений, видео сообщений, обмена файлами, а также посредством игровых вселенных с одновременным вовлечением нескольких игроков \cite{teslenko_cell_network_2018}. Появление возможности создания и распространения мобильных приложений среди пользователей стало новой вехой в развитии методологии полевых экспериментов. Стандартизация процесса взаимодействия пользователей при проведении эксперимента, расширение охвата участников, получение объективных данных, а также более удобное воспроизведение и модификация эксперимента -- являются преимуществами использования мобильных приложений 
\cite{zhang_advantages_2018}. В работе американских ученых университета Калифорнии проведен обзор $101$ статей, в которых использовались мобильные приложения для проведения экспериментов.
Исследование показало, что применение приложений в полевых экспериментах началось ориентировочно с 2013 года. Большая часть приложений ($77\%$) были разработаны в области исследования здравоохранения с 2013 по 2017 год \cite{zhang_advantages_2018}.

Применение мобильных приложений распространено при анализе принятия решений людьми в разных сферах жизни: 
экономике \cite{li_conducting_2021}, здравоохранении \cite{zhang_efficacy_2015, serkh_optimal_2014},
мониторинге эффективности программ обучения \cite{menon_application_2021}, социальных явлениях \cite{chataway_geography_2017} и других. В зависимости от исследования применяют различные подходы к сбору данных. Методы и данные в этом случае классифицируются на качественные и количественные. Выбор методов осуществляется в соответствии с характером исследуемой темы и проблемами исследования. Получение таких данных может быть связано с продолжительностью во времени, при этом объекты исследования успевают существенным образом изменить свои значимые признаки, что соответствует лонгитюдному методу исследования. 

В случае мобильных приложений количественные данные могут быть считаны с различных датчиков на устройстве, таких как термометр, акселерометр, гироскоп, геомагнитный датчик, датчик освещенности, датчик Холла, барометр, гигрометр, педометр, пульсометр, и др. Дополнительная обработанная информация может быть получена о геолокации и текущем адресе на основе GPS, Wi-Fi сетей и базовых станций сотовой сети.

Подход к сбору качественных данных состоит в методе ведения дневника человеком в описательном виде. Информация представляется в виде ежедневного набора записей об активностях и пережитом опыте. Заполнение дневника может быть выполнено за счет видео, аудио-записей, фотографий, текстовых записей, файлов, выбора номинальных категорий состояния и настроения.

Смешанная информация возникает при взаимодействии человека с другим человеком или компьютерной системой посредством мобильного приложения и сети Интернет. Взаимодействие может быть обусловлено как игровой динамикой между оппонентами, так и осуществлением блуждания по контенту приложения (например, поиск информации в веб-браузере, просмотр медиа информации в социальной или новостной сети, и др.). Действия, осуществляемые в приложении пользователем, принятые им решения, местоположение курсора на экране, клики, местоположение взгляда на экране являются примерами цифровой информации, возникающей в ходе работы с приложением. Однако первоначальная природа метода взаимодействия может являться как качественной, так и количественной в зависимости от выбранной человеком стратегии. Так, например, при принятии решений человек может использовать статистический или детерминированный алгоритм определения действия. С другой стороны, применяя качественный подход человек ориентируется на свое настроение, отношение к объекту и мнение о ситуации.

Обладая широкими возможностями по сбору информации о поведении человека, параметрах окружающей среды и позволяющих осуществлять взаимодействие между индивидуумами, мобильные приложения расширили и упростили методы проведения полевых экспериментов.


% \section{Задачи, решаемые в диссертационном исследовании}\label{sec:ch1/sec4}

% Разнообразие процессов случайных блужданий различной природы от бактерий и роботов до цен акций и игр, обуславливает наличие большого числа скрытых факторов, трудных для анализа в эксперименте. Проведенный обзор показал наличие множества сложных экспериментов и соответствующих теоретических моделей, аппроксимирующих поведение объектов на основе части выбранных признаков. В связи с этим для получения всех преимуществ как теоретических основ анализа случайных блужданий, так и оценки соответствия экспериментального проведения наиболее оптимальным стратегиям, необходимо сконструировать процесс, позволяющий собрать достаточное количество статистических данных для проведения анализа и учитывающий особенности влияния скрытых факторов на модель.

% На основании выполненного обзора определим цель работы: получение информации о статистических характеристиках процесса блужданий в ограниченной двухмерной области, индуцированного игровым взаимодействием двух игроков-оппонентов и построение стохастической модели, способной воспроизвести эти характеристики.

% Основные задачи диссертационного исследования:
% \begin{itemize}
%     \item Разработка стохастической модели блужданий на плоскости, управляемых игровым конфликтом, и получение характеристик игрового процесса.
%     \item Разработка и реализация алгоритмов для моделирования игровой пространственной динамики, используя стохастическую модель, получение информации о статистических характеристиках модельного процесса, таких как время достижения границы и его распределение.
%     \item Реализация масштабной серии экспериментов с участием реальных игроков и получение статистически значимого массива данных.
%     \item Исследование взаимосвязи результатов, полученных в ходе теоретического анализа, численного моделирования, и эксперимента, а также их совместная интерпретация.
% \end{itemize}

% \section{Выводы по главе 1}\label{sec:ch1/sec5}

% В первой главе был проведен обзор предметной области и результатов существующих исследований, посвященных игровым взаимодействиям, случайным блужданиям, Марковским цепям, а также теоретическому и экспериментальному способу анализа. Используя проанализированные сведения была предложена новая специальная игровая динамика блужданий на плоскости, управляемых игровым конфликтом двух игроков.

% В главе рассмотрены антагонистические игры на примере задачи о разорении игрока, а также продемонстрирована тесная взаимосвязь задачи с процессами случайных блужданий на отрезке. Приведены работы обобщающие задачи о разорении игрока на многомерный случай нескольких игровых валют. Рассмотрены приложения и задачи, при которых возникают процессы случайных блужданий, а также изучена информация о случайных полетах и выделен класс Марковских случайных процессов. Проанализированы работы связывающие блуждания и игровую динамику, добавляющие случайную компоненту в выбор игроков. Приведены теоретические исследования многошаговых игр на выживание. Рассмотрен метод исследования с применением мобильных приложений для проведения полевых экспериментов.

% % В продолжение работы случайных блужданий игрового типа, автором настоящей диссертационной работы сформулирована новая игровая механика, предлагаемая к исследованию. Приведено описание игры и визуализация мобильного приложения, используемого для проведения полевого эксперимента.

% % На основе проведенного обзора научных работ определена цель диссертационного исследования как получение информации о статистических характеристиках процесса блужданий в ограниченной двухмерной области, индуцированного игровым взаимодействием двух игроков-оппонентов и построение стохастической модели, способной воспроизвести эти характеристики.
