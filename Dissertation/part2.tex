\chapter{Моделирование процессов хемотаксиса бактерий}\label{ch:ch2}

\section{Моделирование процесса генерации степенных распределений на основе механизма дискретного белкового шума}\label{sec:ch2/sec1}
Внутриклеточные сигнальные пути формируют сеть для управления движением бактерии с использованием информации о градиенте концентрации веществ во внеклеточном пространстве. Сеть сигнальных путей процесса хемотаксиса у бактерий E.Coli состоит из небольшого числа компонент, однако этого оказывается достаточно для проявления некоторых свойств сложных биосистем, таких как адаптация и ответ на внешний стимул \cite{korobkova_molecular_2004}. В своей работе авторы показали, что шум, создаваемый сигнальными сетевыми взаимодействиями, контролирует поведенческую вариабельность. Этот механизм демонстрирует свойство биологической системы адаптироваться за счет контроля молекулярного шума. 

Сигнальная сеть хемотаксиса начинается с процесса связывания молекул хемоаттрактанта с сайтами хеморецепторов на цитоплазматической мембране бактерии. Далее, каскад внутриклеточной сигнализации управляет производством белка CheY-P, который диффундирует к моторам и модулирует переключение направления вращения. Изменение направления вращения жгутиков по часовой стрелке (CW) или против часовой стрелки (CCW) заставляет бактерии менять свой паттерн движения: с вращения на одном месте к прямолинейному движению.

Значительный прогресс в понимании статистики переключения двигателей был достигнут с помощью минимальной модели, учитывающей переходы между двумя состояниями через энергетический барьер \cite{tu_how_2005}. Регулирующий путь сводился к действию фосфорилированной формы сигнальной молекулы CheY-P, так что более высокая концентрация CheY-P приводила к более высокой вероятности перехода CCW в CW \cite{khan_steady-state_1980}. В этой работе было обнаружено, что гауссовский шум с конечным временем корреляции может приводить к масштабированию распределений длительностей вращения моторов против часовой стрелки. Подобные флуктуации может вызывать внутренняя стохастичность сигнального генетического пути, в частности, «генетический шум» связанный с конечным числом реагирующих белковых молекул.

Для демонстрации возможности возникновения переключений с промежуточной степенной статистикой в связи с генетическим шумом в данном разделе рассматривается модель химической кинетики синтеза белка CheY-P и возникающего в результате переключения вращения моторов. 

Рассмотрим минимальную модель сигнального пути с точки зрения химической кинетики. Переходы между различными значениями числа молекул белка CheY-P, обозначенного $Y$, следуют следующему уравнению:
\begin{equation}
    \begin{aligned}
        \[Y \mathrel{\mathop{\rightleftarrows}^{\mathrm{K_{y}^{+}}}_{\mathrm{K_{y}^{-}}}} Y + 1\]
    \label{eq:chem}
    \end{aligned}
\end{equation}
где $K_{y}^{+}=\frac{Y_0}/\tau$, $K_{y}^{-}=\frac{Y+1}/\tau$ -- коэффициенты частот перехода между состояниями, $Y_0$ -- равновесное число молекул, характерное время релаксации сигнального пути к равновесному числу молекул. Являясь элементарным процессом рождения-смерти, по своей сути содержит все необходимые свойства: стохастичность, дискретность состояний и конечное время корреляции \tau, которые ранее должны были быть введены аддитивным гауссовским шумом.


\section{Модель движения бактерии с двумя чередующимися поворотными событиями}\label{sec:ch2/sec2}
\section{Скорость смещения бактерий}\label{sec:ch2/sec3}
\section{Численная симуляция движения ансамбля бактерий}\label{sec:ch2/sec4}

