\chapter{Моделирование процессов хемотаксиса бактерий}\label{ch:ch2}

\section{Моделирование процесса генерации степенных распределений на основе механизма дискретного белкового шума}\label{sec:ch2/sec1}
Внутриклеточные сигнальные пути формируют сеть для управления движением бактерии с использованием информации о градиенте концентрации веществ во внеклеточном пространстве. Сеть сигнальных путей процесса хемотаксиса у бактерий E.Coli состоит из небольшого числа компонент, однако этого оказывается достаточно для проявления некоторых свойств сложных биосистем, таких как адаптация и ответ на внешний стимул \cite{korobkova_molecular_2004}. В своей работе авторы показали, что шум, создаваемый сигнальными сетевыми взаимодействиями, контролирует поведенческую вариабельность. Этот механизм демонстрирует свойство биологической системы адаптироваться за счет контроля молекулярного шума. 

Сигнальная сеть хемотаксиса начинается с процесса связывания молекул хемоаттрактанта с сайтами хеморецепторов на цитоплазматической мембране бактерии. Далее, каскад внутриклеточной сигнализации управляет производством белка CheY-P, который диффундирует к моторам и модулирует переключение направления вращения. Изменение направления вращения жгутиков по часовой стрелке (CW) на вращение против часовой стрелки (CCW) заставляет бактерии менять свой паттерн движения: с вращения на одном месте к прямолинейному движению.

Значительный прогресс в понимании статистики переключения двигателей был достигнут с помощью минимальной модели, учитывающей переходы между двумя состояниями через энергетический барьер \cite{tu_how_2005}. Регулирующий путь сводился к действию фосфорилированной формы сигнальной молекулы CheY-P, так что более высокая концентрация CheY-P приводила к более высокой вероятности перехода CCW в CW \cite{khan_steady-state_1980}. В этой работе было обнаружено, что гауссовский шум с конечным временем корреляции может приводить к масштабированию распределений длительностей вращения моторов против часовой стрелки. Подобные флуктуации может вызывать внутренняя стохастичность сигнального генетического пути, в частности, «генетический шум» связанный с конечным числом реагирующих белковых молекул.

Для демонстрации возможности возникновения переключений с промежуточной степенной статистикой в связи с генетическим шумом в данном разделе рассматривается модель химической кинетики синтеза белка CheY-P и возникающего в результате переключения вращения моторов. 

Рассмотрим минимальную модель сигнального пути с точки зрения химической кинетики. Переходы между различными значениями числа молекул белка CheY-P, обозначенного $Y$, задаются следующим уравнением:
\begin{equation}
    \begin{aligned}
        Y \mathrel{\mathop{\rightleftarrows}^{\mathrm{K_{y}^{+}}}_{\mathrm{K_{y}^{-}}}} Y + 1
    \label{eq:chem}
    \end{aligned}
\end{equation}

где $K_{y}^{+}=\frac{Y_0}{\tau}$, $K_{y}^{-}=\frac{Y+1}{\tau}$ -- коэффициенты частот перехода между состояниями, $Y_0$ -- равновесное число молекул, $\tau$ -- характерное время релаксации сигнального пути к равновесному числу молекул. Являясь элементарным процессом рождения-смерти, по своей сути содержит все необходимые свойства: стохастичность, дискретность состояний и конечное время корреляции $\tau$. Значение $Y_0$ можно принять постоянным в связи с тем, что концентрация хемоаттрактанта изменяется медленно (в процессе движения клетки в слабом градиенте или при изменении уровня хемоаттрактанта во времени) по сравнению со шкалой времени переключения. 

По аналогии рассмотрим модель переключения вращения жгутиков. Пусть $X = 0$ соответствует режиму вращения по часовой стрелке, а $X = 1$ — против часовой стрелки:
\begin{equation}
    \begin{aligned}
        X \mathrel{\mathop{\rightleftarrows}^{\mathrm{K_{x}^{+}}}_{\mathrm{K_{x}^{-}}}} X + 1
    \label{eq:turning}
    \end{aligned}
\end{equation}
где коэффициенты $K_x^{+}=K^{+}(1-X)$, $K_x^{-}=K^{-}X$ ограничивают состояния $X={0, 1}$ и переходы контролируются состоянием регулирующего белка CheY-P через соответствующие частоты переключения:
\begin{equation}
    \begin{aligned}
        K^{\pm}=K_0 \exp(\pm\alpha^\pm \cdot \frac{Y_0-Y}{Y_0})
    \label{eq:turning-rates}
    \end{aligned}
\end{equation}
где $\alpha^\pm>0$ характеризует чувствительности частоты переходов, при этом энергетические барьеры аппроксимируются линейной зависимостью относительно уровня CheY-P \cite{khan_steady-state_1980}.

Опишем качественное поведение предложенной модели. Пусть жгутики вращаются по часовой стрелке, что соответствует вращению бактерии на одном месте: $X=0$, а частоты соответственно равны $K_x^{+}=K^{+}$ и $K_x^{-}=0$. При уровне белка CheY-P ниже равновесного состояния $Y<Y_0$ частота переключения в сторону вращения жгутиков против часовой стрелки (то есть прямолинейного движения бактерии) доминирует: $K^{+}>K_0$. Большие значения уровня белка CheY-P $Y>Y_0$, наоборот уменьшает частоту переключений. Соответственно при вращении жгутиков против часовой стрелки смещение уровня белка выше или ниже равновесного состояния приводит к обратному эффекту. Отличающиеся друг од друга уровни интенсивности в свою очередь позволяют независимо настраивать частоты переходов между движение бактерии вперед и вращением на месте.

Однако в рассмотренной модели есть недостаток: частоты переходов могут принимать значения экспоненциально большие или малые в ответ на изменение концентрации $Y$, что может быть биологически неправдоподобным. В соответствии с работой \cite{frankel_adaptability_2014} для учета конечного числа связываний белков CheY-P с моторами, частоты переходов могут быть заменены на:
\begin{equation}
    \begin{aligned}
        K^{\pm}=K_0^{\pm} \exp \left (\pm\frac{\alpha^\pm}{2} \left (\frac{1}{2} - \frac{Y}{Y+K_d} \right ) \right )
    \label{eq:turning-rates-kd}
    \end{aligned}
\end{equation}
Такие коэффициенты соответствуют насыщению с уровнем числа белков, превышающем константу диссоциации $K_d$.

Исследование статистических свойств модели было выполнено с помощью численного моделирования уравнений \cref{eq:chem,eq:turning,eq:turning-rates,eq:turning-rates-kd} с применением стохастического алгоритма Гиллеспи \cite{gillespie_stochastic_2007}. В результате численных расчетов был получен набор реализаций, состоящих из $N=10^7$ шагов, соответствующих одной из возможных химических реакций. Один из типов реакции это переключение направления вращения моторов (уравнение \cref{eq:turning}). Непрерывные участки времени пребывания в каждом состоянии CW и CCW соответственно (между переключениями моторов) обозначим $\{t_{ccw}\}$ и $\{t_{cw}\}$. Всего было собрано не менее $N_{cw} = N_{ccw} = 10^{10}$ отрезков времени пребывания в состояниях CW и CCW.

Полученные выборки были использованы для оценки функций плотности вероятности $p(t_{cw})$ и $p(t_{ccw})$ и проанализированы зависимости от времени релаксации сигнального пути $\tau$ и чувствительности частоты переходов между состояниями $\alpha^{\pm}$. Полученные функции плотности вероятности были аппроксимированы степенной функцией на отрезке $[a, b]$ с использованием линейной регрессии в двойном логарифмическом масштабе методом наименьших квадратов. Качество аппроксимации оценивалось коэффициентом детерминации \cite{magnus_2021}, $R^2 \in [0, 1]$. Отрезок $[a, b]$ выбирался перебором с условиями: длина отрезка не менее $1.3$ декады и коэффициент детерминации $R^2 > 0.98$. Если такой отрезок не был найден, то гипотеза об участке со степенным распределением отвергается. 

Рассмотрим простую форму коэффициентов для частот, заданной уравнением \cref{eq:turning-rates}, и оценим плотности вероятности длительностей для соответственно двух состояний CW (вращение бактерии на одном месте) и CCW (прямолинейное движение). В случае отсутствия молекул CheY-P, $Y=0$ (переключение нечувствительно к CheY-P, $\alpha_{\pm}=0$), процесс является пуассоновским, а плотности вероятности длительностей пребывания в состояниях экспоненциальны, $p(t_{ccw}), p(t_{cw}) \propto \exp(-K_0 t)$.

Далее проводился численный анализ для случая равных чувствительностей переключения $\alpha_\pm = \alpha$. Численные результаты показывают, что коррелированный молекулярный шум дает степенное распределение длительностей, $p(t_{ccw}) \approx t^{-\gamma}$, в четко выраженном интервале (1.5-2 декады) с отсечкой при больших значениях. продолжительность (Рис. \cref{fig:duration-pdf}а).
Плотности с промежуточной степенной асимптотикой на более чем 2 декадах и затухающей с почти экспоненциальной отсечкой, типичны как для измерений in vivo и in vitro пространственной активности бактерий \cite{korobkova_molecular_2004,harris_generalized_2012}.

В работе \cite{tu_how_2005} аналогично было показано, что уменьшение времени корреляции изменяет это распределение в сторону экспоненциального. Рассмотренная модель также способна воспроизводить экспоненциальную статистику длительности CW одновременно со степенным распределением длительностей CCW. При этом, распределения различны из-за несовпадающих чувствительностей частот переходов между состояниями, то есть $\alpha_+ \neq \alpha_-$. Поскольку время пребывания в состоянии определяется частотой покидания этого состояния, то снижение чувствительности перехода из CW в CCW к уровню белка CheY-P должно приводить к экспоненциальному распределению длительностей CW подобно предельному случаю $\alpha^+ = 0$. На рисунке \cref{fig:duration-pdf}б продемонстрирован пример данного режима системы.

Численное моделирование в широком диапазоне параметров показывает, что степенные распределения возникают только тогда, когда время релаксации уровня CheY-P существенно больше, чем время переключения между состояниями CW и CCW $\tau \gg \frac{1}{K_0}$ (Рис. \cref{fig:pdf-gamma-grid-1}a,c). В то же время, увеличение среднего числа сигнальных молекул $Y_0$, приводит к отклонению гипотезы о степенном распределении (Рис. \cref{fig:pdf-gamma-grid-1}а,б). Увеличение числа сигнальных молекул приводит к относительному уменьшению флуктуаций, что в свою очередь уменьшает влияние генетического шума. Параметр чувствительности $\alpha$ контролирует переход между степенным и экспоненциальным распределением длительностей (Рис. \cref{fig:pdf-gamma-grid}а). Значения показателя степени $\gamma$, найденные в большей части области параметров ($1 < \gamma < 3$) согласуются с экспериментальными наблюдениями, которые оценили степенной показатель для кумулятивного распределения длительностей CCW как $\gamma - 1 \approx 1.5$. Это свидетельствуют о том, что медленное метилирование как часть сигнального пути отвечает за длительные временные корреляции в выходном сигнале, что приводит к степенному распределению длительностей \cite{korobkova_molecular_2004}. 

Далее рассмотрим более приемлемую с биологической точки зрения модель скорости перехода в соответствии с уравнением \cref{eq:turning-rates-kd}, которая имеет насыщение в частотах переключения моторов. Для демонстрации крайних случаев эффекта насыщения рассмотрим ситуацию, когда уровень белка CheY-P находится на равновесном уровне $Y = Y_0$ и константа диссоциации меньше значения уровня равновесия $K_d < Y_0$. В Результаты численного моделирования данной модели аналогично подтверждают, что медленная релаксация белка CheY-P вместе с более высокой чувствительностью частоты перехода от CCW к CW к изменению уровня белка приводят к появлению степенной асимптотики длительностей CCW, в то время как длительности CW остаются экспоненциально распределенными (Рис. \cref{fig:pdf-kd}).

Систематическое исследование статистики в зависимости от значений параметров модели представлено на Рис. \cref{fig:pdf-gamma-grid-kd}. На плоскости параметров $(K_d, Y_0)$ выделяются две области, соответствующие двум различным вариантам поведения модели (Рис. \cref{fig:pdf-kd}а). При $Y_0 < K_d$ эффект насыщения имеет слабое влияние, что приводит к сильному отклонению от степенной асимптотики по мере увеличения константы диссоциации. Степенная асимптотика проявляется при достаточно малых показателях $\gamma < 2$. Для $Y_0 > K_d$ введенный эффект насыщения приводит к большим значения показателя $\gamma > 4$ в степенной асимптотике. 

На другой плоскости параметров $(K_d, \alpha_-)$ степенная асимптотика сохраняется в широком диапазоне параметра чувствительности перехода от CCW к CW, $\alpha_-$ (Рис. \cref{fig:pdf-kd}b). Аналогично выделяются режимы с относительно малыми $\gamma < 2$ и большими $\gamma > 4$ значениями показателя степени. Эти режимы обусловлены соотношением между средним числом молекул CheY-P, $Y_0$, и насыщением константы диссоциации $K_d$.

Таким образом биологически релевантная модель дает наиболее схожие с экспериментом показатели степени при коэффициенте диссоциации близком к константе диссоциации $K_d \approx Y_0$ при этом со значениями не превышающими $30$ молекул. Управление параметром $\alpha_-$ в свою очередь позволяет получить степенную асимптотику на большем числе декад, сохранив ее также на меньших промежутках длительностей нахождения в состоянии CCW.

Хотя применение подхода численного моделирования с применением алгоритма Гиллеспи обладает гибкостью к нахождению численных решений произвольных задач, временные затраты на вычисление ограничивают возможности применения модели. Подход full-counting statistics, впервые предложенный в работах по квантовой физики \cite{}, позволяет независимо от природы основного кинетического уравнения оценить статистические свойства дискретной системы. Далее рассмотрим как данный подоход может быть использован для вычисления статистики переключения моторов напрямую из частот перехода $K_x^\pm(Y)$ и $K_y^\pm(Y)$.

Рассмотрим кинетику модели (\cref{eq:chem,eq:turning}) в терминах марковских процессов с непрерывным временем \cite{}. Диаграмма переходов представлены на рис. \cref{fig:transitions}а. Состояние $Z(t)$ процесса в момент времени $t$ представляет собой двумерный вектор $Z = (X, Y)$, где первая компонента представляет направление вращения моторов $X = 0 \equiv \mathrm{CW}$ (по часовой стрелке) и $X = 1 \equiv \mathrm{CCW}$ (против часовой стрелки), а вторая компонента $Y$ -- количество молекул CheY-P. Тогда основное кинетическое уравнение для вектора вероятности $\mathrm{P}(Z, t) \equiv \mathrm{P}(X, Y, t)$ имеет вид:

\begin{equation}
    \begin{aligned}
        &\dot{\mathrm{P}}(\mathrm{CW},0,t)&=&-\left (\frac{Y_0}{\tau} + K_x^+(0) \right ) \mathrm{P}(\mathrm{CW},0,t) + K_x^-(0) \mathrm{P}(\mathrm{CCW},0,t)+&&\\
        &&&+\frac{1}{\tau}\mathrm{P}(\mathrm{CW},1,t),&&\\
        &\dot{\mathrm{P}}(\mathrm{CCW},0,t)&=&-\left (\frac{Y_0}{\tau} + K_x^-(0) \right ) \mathrm{P}(\mathrm{CCW},0,t) + K_x^+(0) \mathrm{P}(\mathrm{CW},0,t)&&\\
        &&&+\frac{1}{\tau}\mathrm{P}(\mathrm{CCW},1,t),&&\\
        &\dots&&\\
        &\dot{\mathrm{P}}(\mathrm{CW},Y,t)&=&-\left (\frac{Y_0+Y}{\tau} + K_x^+(Y) \right ) \mathrm{P}(\mathrm{CW},Y,t) + K_x^-(Y) \mathrm{P}(\mathrm{CCW},Y,t)+&&\\
        &&&+\frac{Y+1}{\tau}\mathrm{P}(\mathrm{CW},Y+1,t)+\frac{Y_0}{\tau}\mathrm{P}(\mathrm{CW},Y-1,t),&&\\
        &\dot{\mathrm{P}}(\mathrm{CCW},Y,t)&=&-\left (\frac{Y_0+Y}{\tau} + K_x^-(Y) \right ) \mathrm{P}(\mathrm{CCW},Y,t) + K_x^+(Y) \mathrm{P}(\mathrm{CW},Y,t)+&&\\
        &&&+\frac{Y+1}{\tau}\mathrm{P}(\mathrm{CCW},Y+1,t)+\frac{Y_0}{\tau}\mathrm{P}(\mathrm{CCW},Y-1,t),&&\\
    \end{aligned}
    \label{eq:master-transitions}
\end{equation}

Примечательно, что наша модель служит обобщением жидкостной очереди, управляемой процессом рождения-смерти, предложенным ван Доорном, Ягерсом и де Витом \cite{}. В терминах марковских процессов реакция \cref{eq:turning} представляет собой обмен вероятностью между двумя состояниями. Этот обмен можно рассматривать как обмен фиксированного количества несжимаемой жидкости между двумя баками со скоростями, определяемыми состоянием процесса рождения-смерти. В исходной постановке задачи был только один бак бесконечного объема и неограниченное количество жидкости. Длительность в состоянии CCW -- это время, которое одна молекула жидкости проводит в резервуаре <<1>>, прежде чем покинуть его. Вводя вектор вероятности 

\begin{equation}
    P(t) = \begin{bmatrix}
    P(X = \mathrm{CW}, Y = 0; t)\\
    P(X = \mathrm{CW}, Y = 1; t)\\
    P(X = \mathrm{CW}, Y = 2; t)\\
    ...\\
    P(X = \mathrm{CCW}, Y = 0; t)\\
    P(X = \mathrm{CCW}, Y = 1; t)\\
    P(X = \mathrm{CCW}, Y = 2; t)\\
    ...\\
    \end{bmatrix}
\label{eq:state-probs}
\end{equation}

и матрицу чатот переходов $\boldsymbol{\mathrm{Q}}$ \cite{}, мы можем переписать уравнение \cref{eq:master-transitions} в компактной форме: 

\begin{equation}
    \mathrm{P}(t) = \boldsymbol{\mathrm{Q}} \mathrm{P}(t)
    \label{eq:master-transitions-compact}
\end{equation}

Матрица $\boldsymbol{\mathrm{Q}}$ состоит из двух одинаковых полубесконечных диагональных блоков $\boldsymbol{\mathrm{Q}}_{BD}$ (матриц трехдиагонального вида, состоящих из частот переходов процесса рождения-гибели), состоящих из блоков $\boldsymbol{\mathrm{Q}}_{B}$ 

\begin{equation}
    \boldsymbol{\mathrm{Q}}_{B} = 
    \begin{bmatrix} 0&K^+_x (Y)\\K^-_x (Y)&0 \end{bmatrix}
    \label{eq:transition-block}
\end{equation}

\section{Модель движения бактерии с двумя чередующимися поворотными событиями}\label{sec:ch2/sec2}
\section{Скорость смещения бактерий}\label{sec:ch2/sec3}
\section{Численная симуляция движения ансамбля бактерий}\label{sec:ch2/sec4}

