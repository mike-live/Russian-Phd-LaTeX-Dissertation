\chapter{Моделирование процессов хемотаксиса бактерий}\label{ch:ch2}

\section{Моделирование процесса генерации степенных распределений на основе механизма дискретного белкового шума}\label{sec:ch2/sec1}
Внутриклеточные сигнальные пути формируют сеть для управления движением бактерии с использованием информации о градиенте концентрации веществ во внеклеточном пространстве. Сеть сигнальных путей процесса хемотаксиса у бактерий E.Coli состоит из небольшого числа компонент, однако этого оказывается достаточно для проявления некоторых свойств сложных биосистем, таких как адаптация и ответ на внешний стимул \cite{korobkova_molecular_2004}. В своей работе авторы показали, что шум, создаваемый сигнальными сетевыми взаимодействиями, контролирует поведенческую вариабельность. Этот механизм демонстрирует свойство биологической системы адаптироваться за счет контроля молекулярного шума. 

Сигнальная сеть хемотаксиса начинается с процесса связывания молекул хемоаттрактанта с сайтами хеморецепторов на цитоплазматической мембране бактерии. Далее, каскад внутриклеточной сигнализации управляет производством белка CheY-P, который диффундирует к моторам и модулирует переключение направления вращения. Изменение направления вращения жгутиков по часовой стрелке (CW) на вращение против часовой стрелки (CCW) заставляет бактерии менять свой паттерн движения: с вращения на одном месте к прямолинейному движению.

Значительный прогресс в понимании статистики переключения двигателей был достигнут с помощью минимальной модели, учитывающей переходы между двумя состояниями через энергетический барьер \cite{tu_how_2005}. Регулирующий путь сводился к действию фосфорилированной формы сигнальной молекулы CheY-P, так что более высокая концентрация CheY-P приводила к более высокой вероятности перехода CCW в CW \cite{khan_steady-state_1980}. В этой работе было обнаружено, что гауссовский шум с конечным временем корреляции может приводить к масштабированию распределений длительностей вращения моторов против часовой стрелки. Подобные флуктуации может вызывать внутренняя стохастичность сигнального генетического пути, в частности, «генетический шум» связанный с конечным числом реагирующих белковых молекул.

Для демонстрации возможности возникновения переключений с промежуточной степенной статистикой в связи с генетическим шумом в данном разделе рассматривается модель химической кинетики синтеза белка CheY-P и возникающего в результате переключения вращения моторов. 

Рассмотрим минимальную модель сигнального пути с точки зрения химической кинетики. Переходы между различными значениями числа молекул белка CheY-P, обозначенного $Y$, задаются следующим уравнением:
\begin{equation}
    \begin{aligned}
        Y \mathrel{\mathop{\rightleftarrows}^{\mathrm{K_{y}^{+}}}_{\mathrm{K_{y}^{-}}}} Y + 1
    \label{eq:chem}
    \end{aligned}
\end{equation}

где $K_{y}^{+}=\frac{Y_0}{\tau}$, $K_{y}^{-}=\frac{Y+1}{\tau}$ -- коэффициенты частот перехода между состояниями, $Y_0$ -- равновесное число молекул, $\tau$ -- характерное время релаксации сигнального пути к равновесному числу молекул. Являясь элементарным процессом рождения-смерти, по своей сути содержит все необходимые свойства: стохастичность, дискретность состояний и конечное время корреляции $\tau$. Значение $Y_0$ можно принять постоянным в связи с тем, что концентрация хемоаттрактанта изменяется медленно (в процессе движения клетки в слабом градиенте или при изменении уровня хемоаттрактанта во времени) по сравнению со шкалой времени переключения. 

По аналогии рассмотрим модель переключения вращения жгутиков. Пусть $X = 0$ соответствует режиму вращения по часовой стрелке, а $X = 1$ — против часовой стрелки:
\begin{equation}
    \begin{aligned}
        X \mathrel{\mathop{\rightleftarrows}^{\mathrm{K_{x}^{+}}}_{\mathrm{K_{x}^{-}}}} X + 1
    \label{eq:turning}
    \end{aligned}
\end{equation}
где коэффициенты $K_x^{+}=K^{+}(1-X)$, $K_x^{-}=K^{-}X$ ограничивают состояния $X={0, 1}$ и переходы контролируются состоянием регулирующего белка CheY-P через соответствующие частоты переключения:
\begin{equation}
    \begin{aligned}
        K^{\pm}=K_0 \exp(\pm\alpha^\pm \cdot \frac{Y_0-Y}{Y_0})
    \label{eq:turning-rates}
    \end{aligned}
\end{equation}
где $\alpha^\pm>0$ характеризует чувствительности частоты переходов, при этом энергетические барьеры аппроксимируются линейной зависимостью относительно уровня CheY-P \cite{khan_steady-state_1980}.

Опишем качественное поведение предложенной модели. Пусть жгутики вращаются по часовой стрелке, что соответствует вращению бактерии на одном месте: $X=0$, а частоты соответственно равны $K_x^{+}=K^{+}$ и $K_x^{-}=0$. При уровне белка CheY-P ниже равновесного состояния $Y<Y_0$ частота переключения в сторону вращения жгутиков против часовой стрелки (то есть прямолинейного движения бактерии) доминирует: $K^{+}>K_0$. Большие значения уровня белка CheY-P $Y>Y_0$, наоборот уменьшает частоту переключений. Соответственно при вращении жгутиков против часовой стрелки смещение уровня белка выше или ниже равновесного состояния приводит к обратному эффекту. Отличающиеся друг од друга уровни интенсивности в свою очередь позволяют независимо настраивать частоты переходов между движение бактерии вперед и вращением на месте.

Однако в рассмотренной модели есть недостаток: частоты переходов могут принимать значения экспоненциально большие или малые в ответ на изменение концентрации $Y$, что может быть биологически неправдоподобным. В соответствии с работой \cite{frankel_adaptability_2014} для учета конечного числа связываний белков CheY-P с моторами, частоты переходов могут быть заменены на:
\begin{equation}
    \begin{aligned}
        K^{\pm}=K_0^{\pm} \exp \left (\pm\frac{\alpha^\pm}{2} \left (\frac{1}{2} - \frac{Y}{Y+K_d} \right ) \right)
    \label{eq:turning-rates-kd}
    \end{aligned}
\end{equation}
Такие коэффициенты соответствуют насыщению с уровнем числа белков, превышающем константу диссоциации $K_d$.

Исследование статистических свойств модели было выполнено с помощью численного моделирования уравнений \cref{eq:chem,eq:turning,eq:turning-rates,eq:turning-rates-kd} с применением стохастического алгоритма Гиллеспи \cite{gillespie_stochastic_2007}. В результате численных расчетов был получен набор реализаций, состоящих из $N=10^7$ шагов, соответствующих одной из возможных химических реакций. Один из типов реакции это переключение направления вращения моторов (уравнение \cref{eq:turning}). Непрерывные участки времени пребывания в каждом состоянии CW и CCW соответственно (между переключениями моторов) обозначим $\{t_{ccw}\}$ и $\{t_{cw}\}$. Всего было собрано не менее $N_{cw} = N_{ccw} = 10^{10}$ отрезков времени пребывания в состояниях CW и CCW.

Полученные выборки были использованы для оценки функций плотности вероятности $p(t_{cw})$ и $p(t_{ccw})$ и проанализированы зависимости от времени релаксации сигнального пути $\tau$ и чувствительности частоты переходов между состояниями $\alpha^{\pm}$. Полученные функции плотности вероятности были аппроксимированы степенной функцией на отрезке $[a, b]$ с использованием линейной регрессии в двойном логарифмическом масштабе методом наименьших квадратов. Качество аппроксимации оценивалось коэффициентом детерминации \cite{}, $R^2 \in [0, 1]$. Отрезок $[a, b]$ выбирался перебором с условиями: длина отрезка не менее $1.3$ декады и коэффициент детерминации $R^2 > 0.98$. Если такой отрезок не был найден, то гипотеза об участке со степенным распределением отвергается. 

Рассмотрим простую форму коэффициентов для частот, заданной уравнением \cref{eq:turning-rates}, и оценим плотности вероятности для соответственно двух состояний CW (вращение бактерии на одном месте) и CCW (прямолинейное движение). В случае отсутствия молекул CheY-P, $Y_0 = 0$ (переключение нечувствительно к CheY-P, ± = 0), процесс является пуассоновским, а PDF интервалов пребывания экспоненциальны, p(tccw), p(tcw) exp(−K0t).

\section{Модель движения бактерии с двумя чередующимися поворотными событиями}\label{sec:ch2/sec2}
\section{Скорость смещения бактерий}\label{sec:ch2/sec3}
\section{Численная симуляция движения ансамбля бактерий}\label{sec:ch2/sec4}

