Применяя символьные вычисления в пакете MAPLE \cite{monagan2012maple}, тем же способом было найдено решение с произвольными вероятностями перехода в 4 направлениях на четырех-связной решетке. 

Аналогичным образом было найдено решение с произвольными вероятностями перехода в 4 направлениях на четырех-связной решетке на основе символьных вычислений в пакете MAPLE \cite{monagan2012maple}. 

------------------------------------------------------------------------------------------------------------------

Развитие вычислительной техники и рост производительности позволил решать задачи не только на бумаге, проводя сложный анализ функциональных уравнений, но и применяя принципы численного моделирования и симуляции Монте-Карло \cite{Sobol_monte-carlo_1968} для решения задач фиксированной размерности с конкретными значениями параметров.

Развитие вычислительной техники и рост производительности позволил решать задачи не только на бумаге, проводя сложный анализ функциональных уравнений, но и применяя принципы численного моделирования и симуляции Монте-Карло \cite{Sobol_monte-carlo_1968} для решения задач фиксированной размерности с конкретными значениями параметров.


------------------------------------------------------------------------------------------------------------------

Используя свойство обращения интеграла от функции отклика в ноль \cref{eq:response-ecoli-zero} и линейную функцию концентрации \cref{eq:linear-concentration}, подстановка функции отклика \cref{eq:response-ecoli} в частоту переключений \cref{eq:turning-frequency} дает следующий результат:

Подстановка функции отклика \cref{eq:response-ecoli} в частоту переключений \cref{eq:turning-frequency} с учетом свойства обращения интеграла от функции отклика в ноль \cref{eq:response-ecoli-zero} и линейной функции концентрации \cref{eq:linear-concentration} дает следующий результат:

------------------------------------------------------------------------------------------------------------------

Решение для случая двух углов с фиксированным вторым углом, равным $90^\circ$, полученное в работе Йоханнеса Тактикоса \cite{taktikos_how_2013}, формула №(28) также является частным случаем приведенного решения для $\beta=0$:

Решение для случая двух углов с фиксированным вторым углом, равным $90^\circ$, полученное в работе Йоханнеса Тактикоса \cite{taktikos_how_2013}, формула №(28), также является частным случаем приведенного решения для $\beta=0$:

------------------------------------------------------------------------------------------------------------------

По рассмотренному ранее алгоритму движения бактерий производится $K$ итераций с фиксированным шагом $\Delta t$, применяя необходимый паттерн движения.

Выполняется $K$ итераций рассмотренного ранее алгоритма движения бактерии с фиксированным шагом $\Delta t$ и необходимым паттерном движения.

------------------------------------------------------------------------------------------------------------------

Оценивая наклон линейной регрессии для полученной зависимости среднего квадратичного отклонения от времени, были получены значения коэффициентов диффузии для E.~coli $D_{EC}=116.9 \frac{\textrm{мкм}^2}{\textrm{с}}$, для V.~alginolyticus $D_{VA}=96.8 \frac{\textrm{мкм}^2}{\textrm{с}}$.


Коэффициенты диффузии были вычислены с применением формулы \cref{eq:msd} и линейной регрессии для зависимости среднего квадратичного отклонения от времени, полученной в численном эксперименте. Соответствующие значения коэффициентов диффузии: для E.~coli $D_{EC}=116.9 \frac{\textrm{мкм}^2}{\textrm{с}}$, для V.~alginolyticus $D_{VA}=96.8 \frac{\textrm{мкм}^2}{\textrm{с}}$.

------------------------------------------------------------------------------------------------------------------

Продолжая исследования И.~В.~Романовского \cite{romanovsky_1961}, обобщившего игры на выживание до многомерного случая и сформулировавшего в самом общем виде как управляемое игрой блуждание на конечной области с границей, в настоящей диссертационной работе предложена игровая механика взаимодействия двух оппонентов, управляющих блужданием фишки на поле.

В продолжение исследований И.~В.~Романовского \cite{romanovsky_1961}, обобщившего игры на выживание до многомерного случая и сформулировавшего их в самом общем виде как управляемое игрой блуждание на конечной области с границей, в настоящей диссертационной работе предложена игровая механика взаимодействия двух оппонентов, управляющих блужданием фишки на поле.

------------------------------------------------------------------------------------------------------------------

Применяя многократно матрицу переходов для внутренних состояний и проводя суммирование результирующей матрицы по всем моментам времени, теория поглощающих Марковских цепей вводит понятие фундаментальной матрицы $N$.

***
В теории поглощающих Марковских цепей вводится понятие фундаментальной матрицы $N$, определяемое как сумма многократно примененных матриц переходов для внутренних состояний.

------------------------------------------------------------------------------------------------------------------

Применяя метод расчета фундаментальной матрицы для поглощающей Марковской цепи для разных размеров поля, были получены значения среднего времени игры для случая BvB. Используя аппроксимацию квадратичной зависимостью на основе метода наименьших квадратов (МНК) зависимости среднего времени для размеров полей в диапазоне от $3$ до $1001$, были получены численные коэффициенты параболы: $\boldsymbol{\mathsf{t_n^{BvB}}} = 0$.


С использованием метода расчета фундаментальной матрицы для поглощающей Марковской цепи для различных размеров поля, были получены значения среднего времени игры для случая BvB. Результирующая кривая зависимости среднего времени для размеров полей в диапазоне от $3$ до $1001$ была аппроксимирована квадратичной функцией на основе метода наименьших квадратов (МНК). В результате аппроксимации были получены численные коэффициенты параболы: $\boldsymbol{\mathsf{t_n^{BvB}}} = 0$.

------------------------------------------------------------------------------------------------------------------

Анализируя стратегии и значения среднего времени игры, найденные с применением глобальной оптимизации, для малых размерностей, а также учитывая свойства игры, для достижения оптимальных значений были предложены $2$ стратегии, опубликованные в тезисах конференции \cite{confbib1}.


Информация о стратегиях и значениях среднего времени игры, найденная с применением глобальной оптимизации для малых размерностей, а также о свойствах игры, была использована для обнаружения двух оптимальных стратегий, опубликованных в тезисах конференции \cite{confbib1}.


------------------------------------------------------------------------------------------------------------------

Применяя символьные вычисления Wolfram Mathematica в методе расчета среднего времени игры с использованием фундаментальной матрицы было продемонстрировано для размеров поля $5, 7, 9, 11$, что все переменные $f_{ij}^A$, соответствующие состояниям не инцидентным границе, после упрощения выражения имеют нулевые коэффициенты.

С использованием символьных вычислений Wolfram Mathematica и метода расчета среднего времени игры, для размеров поля $5, 7, 9, 11$ было продемонстрировано, что все переменные $f_{ij}^A$, соответствующие состояниям не инцидентным границе, после упрощения выражения имеют нулевые коэффициенты.

------------------------------------------------------------------------------------------------------------------

Используя когнитивные тесты, предложенные исследователями кафедры нейропсихофизиологии ННГУ им.~Н.И~Лобачевского, а также экспериментальные и организационные возможности ИББМ ННГУ им.~Н.И.~Лобачевского, был получен набор данных с информацией о результатах когнитивных тестах, возрасте и поле испытуемых.

Сбор данных с информацией о результатах когнитивных тестов, возрасте и поле испытуемых, был проведен с использованием экспериментальных и организационных возможностей ИББМ ННГУ им.~Н.И.~Лобачевского. Используемые когнитивные тесты были предложены исследователями кафедры нейропсихофизиологии ННГУ им.~Н.И~Лобачевского.

------------------------------------------------------------------------------------------------------------------

Используя корреляционный анализ, 64 показателя были упорядочены по уровню корреляции с возрастом.

Статистические когнитивные показатели были упорядочены по уровню корреляции с возрастом.