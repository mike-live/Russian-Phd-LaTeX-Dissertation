\pdfbookmark{Общая характеристика работы}{characteristic}             % Закладка pdf
\section*{Общая характеристика работы}

\newcommand{\actuality}{\pdfbookmark[1]{Актуальность}{actuality}\underline{\textbf{\actualityTXT}}}
\newcommand{\progress}{\pdfbookmark[1]{Разработанность темы}{progress}\underline{\textbf{\progressTXT}}}
\newcommand{\aim}{\pdfbookmark[1]{Цели}{aim}\underline{{\textbf\aimTXT}}}
\newcommand{\tasks}{\pdfbookmark[1]{Задачи}{tasks}\underline{\textbf{\tasksTXT}}}
\newcommand{\aimtasks}{\pdfbookmark[1]{Цели и задачи}{aimtasks}\aimtasksTXT}
\newcommand{\novelty}{\pdfbookmark[1]{Научная новизна}{novelty}\underline{\textbf{\noveltyTXT}}}
\newcommand{\influence}{\pdfbookmark[1]{Практическая значимость}{influence}\underline{\textbf{\influenceTXT}}}
\newcommand{\methods}{\pdfbookmark[1]{Методология и методы исследования}{methods}\underline{\textbf{\methodsTXT}}}
\newcommand{\defpositions}{\pdfbookmark[1]{Положения, выносимые на защиту}{defpositions}\underline{\textbf{\defpositionsTXT}}}
\newcommand{\reliability}{\pdfbookmark[1]{Достоверность}{reliability}\underline{\textbf{\reliabilityTXT}}}
\newcommand{\probation}{\pdfbookmark[1]{Апробация}{probation}\underline{\textbf{\probationTXT}}}
\newcommand{\contribution}{\pdfbookmark[1]{Личный вклад}{contribution}\underline{\textbf{\contributionTXT}}}
\newcommand{\publications}{\pdfbookmark[1]{Публикации}{publications}\underline{\textbf{\publicationsTXT}}}


{\actuality} 

Существование (и выживание) многих биологических видов возможно благодаря таксису, то есть способности живых организмов двигаться в направлении максимальной концентрации необходимых для их жизнедеятельности ресурсов (пищи, воды, химических агентов) и возрастающей оптимальности условий существования (нахождение области с наиболее приемлемой комбинацией температуры, влажности, освещенности и другими факторами). Очевидно, что, чем лучше стратегия таксиса, используемая организмом, тем выше его индивидуальная выживаемость и эволюционные преимущества его вида в целом.

Примером таксиса является хемотаксис -- движение бактерии в направлении максимальной концентрации химического аттрактанта. Линейные размеры бактерии слишком малы, чтобы оценивать градиент как разницу концентраций измеренных в разных частях клетки. Вместо этого бактерия использует механизм памяти, <<интегрирующий>> измеренную концентрацию во времени, что позволяет бактерии, используя стратегию случайных поворотов, частота которых зависит от предыстории изменения концентрации, двигаться в направлении источника аттрактанта. На этом примере уже видно, насколько сложными являются стратегии таксиса, используемыми живыми организмами. В настоящей работе проводится анализ стратегий движения, управляемых двумя чередующимися углами, имитирующих движение таких видов бактерий как V.~alginolyticus, что является расширением существующей теории, учитывающей единственный угол поворота. 

Управление хемотаксисом бактерии осуществляется за счет генетической сети, контролирующей смену режимов частот между стадиями движения. Экспериментальные данные демонстрируют наличие как экспоненциальных распределений длительностей, так и возникновение степенных асимптотик в распределении на некотором участке. Построение простой модели, объясняющей механизмы управления частотой поворотов в паттернах движения бактерий, все еще остается открытой проблемой.

Получение локальных сигналов, указывающих на расположение источника, не всегда возможно в макроскопических масштабах, потому что динамика среды, в которых движется организм (воздух, жидкость), привносит существенный фактор помех и случайных флуктуаций концентраций вещества. Таким образом, животные, ощущающие запахи в воздухе или воде, обнаруживают их только эпизодически, как <<пятна>> концентрации, постоянно изменяемые ветром или турбулентным течением. Организм -- природный или искусственный (как, например, автономный робот) -- нуждается в оптимальной стратегии движения, использующей спорадические сигналы и частичную информацию.

Такие стратегии получили название <<инфотаксиса>>. Процесс поиска можно рассматривать как получение информации об исходном местоположении. Информация в данном процессе играет роль аналогичную концентрации в хемотаксисе. Оптимальные стратегии инфотаксиса локально максимизируют ожидаемую скорость получения информации. Существующие алгоритмы инфотаксиса приводят к траекториям, которые характеризуются <<зигзагообразной>> структурой, подобной той, которая наблюдается в полете насекомых. Использование систем из множества агентов имеет больше преимуществ, чем одиночный робот при локализации источника за счет возможности уменьшения ожидаемого времени поиска и уменьшения вероятности попадания в локальные экстремумы.

Процессы таксиса могут быть обусловлены влиянием окружающей среды, имеющей как случайную компоненту, так и целенаправленную компоненту, в качестве которой может выступать другой вид организмов, конфликтующий за ресурсы и условия существования. В настоящей работе проводится анализ влияния случайной компоненты в рамках анализа стратегий хемотаксиса бактерий, а также исследование компоненты, связанной с возникающим конфликтом противоборствующих сторон, в рамках предложенной модели игровых случайных блужданий. 

Игровые взаимодействия, в которых каждый участник руководствуется принципами максимизации своего личного выигрыша или целого коллектива в некоторой игре -- существенно отличны от обычных физических взаимодействий. Дополненные механизмами обучения, оценки и адаптации, игровые взаимодействия определяют принципиально новый тип динамики. К настоящему времени стало понятно, что область применения игровой динамики гораздо шире социологии и экологии, и что соответствующие методы могут быть использованы для моделирования динамики финансовых рынков и интерпретации процессов Бозе-Эйнштейн конденсации в открытых квантовых системах. 

В качестве модельного процесса в настоящей работе предложена игра, в которой оппоненты управляют перемещением агента на квадратной решетке, делая независимый выбор одной из двух возможных стратегий на каждом шаге игры. Информация о возможных перемещениях агента, открыта для обоих игроков и организована в виде матрицы. Целью первого игрока является максимально долгое удержание агента внутри ограниченной области, а второго -- максимально быстрое достижение им поглощающей границы. Результатом игры является время поглощения. 

Игры такого типа были определены в некоторых работах по теории игр и принятия решений еще в 1950-1960х годах. Однако, в силу вычислительной сложности задачи, количественные результаты были получены только для очень простых моделей с тремя-пятью состояниями, что не позволяет делать заключения об асимптотических характеристиках процесса и использовать для их анализа статистический подход. Существует тесная связь между <<играми на выживание>> и процессами случайных блужданий, например, классическая задача о разорении игрока допускает прямую интерпретацию как процесс случайного блуждания на конечном интервале. 

С другой стороны, проблема случайных блужданий в ограниченной области и такие вопросы, как оценка среднего времени достижения границы (или любого другого определенного региона), в настоящее время испытывает очередную волну интереса, вызванную перспективами приложения этого подхода к проблемам молекулярной биологии, химической кинетики, и экологии. 

Идея данной работы состоит в том, что игровые блуждания в конечных пространственных областях, ограниченных поглощающими границами, могут быть исследованы и квантифицированы, используя методологию теории случайных блужданий и теории игр. Данная работа предполагает проведение масштабного полевого эксперимента с участием реальных игроков. Таким образом, область исследований лежит на стыке нескольких областей: теории случайных блужданий, теории хемотаксиса, теории игр и технологии социальных экспериментов. 

% В соответствии с паспортом специальности 01.04.03 <<Радиофизика>>, данная диссертация
% относится к области исследований: <<4. Исследование флуктуаций, шумов, случайных процессов и полей в сосредоточенных и распределенных стохастических системах (статистическая радиофизика). Создание новых методов анализа и статистической обработки сигналов в условиях помех. Разработка статистических основ передачи информации. Исследование нелинейной динамики, пространственно-временного хаоса и самоорганизации в неравновесных физических, биологических, химических и экономических системах.>>, <<4. Реализация эффективных численных методов и алгоритмов в виде
% комплексов проблемно-ориентированных программ для проведения
% вычислительного эксперимента>>.

\ifsynopsis
% Этот абзац появляется только в~автореферате.
% Для формирования блоков, которые будут обрабатываться только в~автореферате,
% заведена проверка условия \verb!\!\verb!ifsynopsis!.
% Значение условия задается в~основном файле документа (\verb!synopsis.tex! для
% автореферата).
\else
%Этот абзац появляется только в~диссертации.
%Через проверку условия \verb!\!\verb!ifsynopsis!, задаваемого в~основном файле
%документа (\verb!dissertation.tex! для диссертации), можно сделать новую
%команду, обеспечивающую появление цитаты в~диссертации, но~не~в~автореферате.
\fi

% {\progress}
% Этот раздел должен быть отдельным структурным элементом по
% ГОСТ, но он, как правило, включается в описание актуальности
% темы. Нужен он отдельным структурынм элемементом или нет ---
% смотрите другие диссертации вашего совета, скорее всего не нужен.

{\aim} данной работы является развитие теории анализа стратегий и механизмов таксиса на примере хемотаксиса бактерий, а также построение и развитие теории анализа статистических свойств и оптимальных стратегий игровой модели блужданий в ограниченной двухмерной области, индуцированного игровым взаимодействием двух игроков-оппонентов, подтвержденной экспериментальными данными и численными экспериментами.

Для~достижения поставленной цели необходимо было решить следующие {\tasks} диссертационного исследования:
\begin{enumerate}[beginpenalty=10000] % https://tex.stackexchange.com/a/476052/104425
    \item Получить аналитическую форму средней скорости колонии бактерий в случае паттерна движения с двумя чередующимися углами и подтвердить ее в численном эксперименте при малом химическом градиенте.
    \item Построить и проанализировать стохастическую модель влияния белкового шума на частоту переключения вращения моторов у бактерий. Получить способ оценки распределения длительностей для модели и подобрать параметры для генерации двух режимов: степенное распределение и экспоненциальное распределение длительностей.
    \item Разработать стохастическую модель блужданий на плоскости, управляемых игровым конфликтом, получить статистические характеристики игрового процесса, провести анализ стратегий и найти оптимальные стратегии в предложенной игре. 
    \item Провести масштабную серию экспериментов с участием реальных игроков, получить статистически значимый массив данных и провести анализ соответствия эксперимента и модели.
\end{enumerate}


{\novelty}
В работе получены следующие новые научные результаты:
\begin{enumerate}[beginpenalty=10000] % https://tex.stackexchange.com/a/476052/104425
  \item Получена аналитическая формула для расчета средней скорости колонии бактерий в случае паттерна движения с двумя чередующимися углами. Впервые предложен подход для нахождения средней скорости колонии бактерий при стратегии движения с произвольным конечным количеством чередующихся поворотов методом решения системы линейных уравнений.
  \item Впервые предложена математическая модель генной сети, генерирующей степенное распределение длительностей вращения жгутиковых моторов за счет белкового шума. Продемонстрировано применение подхода нахождения распределения длительностей к предложенной математической модели.
  \item Впервые предложен игровой конфликт двух игроков, управляющих блужданием фишки на плоскости, реализованный в виде мобильного приложения. Новизна подхода заключается в применении интернет-технологий для реализации игры одновременно учитывающей как возможность создания игрового взаимодействия между игроками, так и процесса случайного блуждания.
  \item Впервые разработана стохастическая модель предложенной игровой динамики, позволяющая получить характеристики игрового процесса и воспроизвести результаты масштабного эксперимента игр полученных реальными игроками. Новизна подхода состоит в возможности исследования модельного процесса случайного блуждания, позволяющего воспроизвести экспериментально полученные характеристики, а также провести точное сравнение модели с полевым экспериментом.
  \item Впервые разработаны и реализованы численные методы для расчета статистических характеристик игрового процесса при фиксированных заданных стратегиях игроков, таких как среднее время игры, распределение времен игры, распределение вероятностей наблюдения фишки в состояниях конечной решетки. 
  \item Впервые найдены оптимальные средние времена для трех случаев игры, предложены классы оптимальных стратегий и визуализированы конкретные стратегии. Предложен подход для нахождения оптимальных стратегий при произвольной стратегии оппонента.
\end{enumerate}

{\influence} Исследование таксиса сопряжено с поиском и анализом стратегий наиболее применимых в условиях возникающих спорадических сигналов и частичной информации. Результаты работы могут иметь применимость как в конкретных сферах -- редактирование генома бактерий для достижения наиболее эффективной таргетной доставки лекарств в организме, так и в сферах управления искусственными мультиагентными системами при решении задачи поиска некоторого источника. Рассмотрение предложенной модели игровых блужданий является отправной точкой для теоретических исследований конфликта между мультиагентными системами и между агентами и средой.

{\methods} 
В работе используются методы математического моделирования, статистической физики, теории игр, теории вероятностей, теории марковских цепей, теории случайных процессов, теории случайных блужданий и численного моделирования. Дополнительно используется подход к проведению масштабного полевого эксперимента с применением интернет-технологий и мобильных приложений.


{\defpositions}
\begin{enumerate}[beginpenalty=10000] % https://tex.stackexchange.com/a/476052/104425
    \item Определена аналитическая форма средней скорости движения колонии бактерий в условиях малого линейного градиента химического вещества для паттерна движения, характеризующегося двумя чередующимися углами. Продемонстрирована возможность получения аналогичного результата для стратегии движения с произвольным конечным количеством чередующихся поворотов методом решения системы линейных уравнений.
    \item Предложена простая математическая модель генной сети, генерирующей степенное распределение длительностей вращения жгутиковых моторов за счет белкового шума (дробового шума). Развит математический аппарат для определения результирующего распределения длительностей пребывания в одном из двух состояний при наличии скрытой переменной. 
    \item Построена стохастическая модель, описывающая игровой конфликт двух игроков, управляющих блужданием фишки на конечной квадратной решетке. Созданы методы для расчета статистических характеристик игрового процесса при фиксированных заданных стратегиях игроков, таких как среднее время игры, распределение времен игры, распределение вероятностей наблюдения фишки в состояниях конечной решетки. 
    \item Развит математический аппарат нахождения оптимальных стратегий для игры двух оппонентов, управляющих блужданием фишки на конечной квадратной решетке. Получены оптимальные средние времена для двух случаев игры против стратегии случайного равновероятного выбора, и установлена оптимальность данных значений для малых размеров игрового поля, а также найден класс стратегий, достигающих оптимальной оценки. Определены оптимальные стратегии игроков в случае полноценного игрового конфликта и вычислены соответствующие средние оптимальные времена.
    \item Обнаружено соответствие эксперимента предложенной модели игры и установлены возникающие особенности синхронизации игроков, а также выявлены отличия стратегий игроков от предполагаемых оптимальных стратегий на основе сравнения статистических свойств траекторий и стратегий участников масштабного эксперимента, проведенного с применением мобильных и интернет-технологий.
\end{enumerate}

{\reliability} полученных результатов, научных положений и выводов, полученных в диссертации, обеспечивается корректным обоснованием постановок задач, точной формулировкой критериев, подтверждается результатами вычислительных экспериментов по использованию предложенных в диссертации методов и алгоритмов, сравнением полученных результатов с проведенными ранее исследованиями и перекрестной проверкой с применением различных методов. Результаты находятся в соответствии с результатами, полученными другими авторами.

{\probation}
Основные результаты диссертационного исследования были представлены на следующих научных конференциях и фестивалях в 2017-2023 гг. в форме секционных и стендовых докладов:
\begin{itemize}
    \item XXVI научная конференция по радиофизике, посвященная 120-летию со дня рождения М.Т. Греховой. (Нижний Новгород, ННГУ им.~Н.И.~Лобачевского, 2022);
    \item Всероссийский фестиваль молодежных инноваций Иннофест. (Нижний Новгород, ННГУ им.~Н.И.~Лобачевского, 2020);
    \item 74-я всероссийская с международным участием школа-конференция молодых ученых, посвященная памяти проф. А.П. Веселова <<Биосистемы: организация, поведение, управление>>. (Нижний Новгород, ННГУ им.~Н.И.~Лобачевского, 2021);
    \item 3rd International Conference Volga Neuroscience Meeting 2021. (Нижний Новгород, Отель <<Чайка>>, 2021);
    \item Нелинейные дни в Саратове для молодых: сборник научных трудов. (Саратов, СГУ им.~Н.Г.~Чернышевского, 2023)
    \item 71-я всероссийская с международным участием школа-конференция молодых ученых <<Биосистемы: организация, поведение, управление>>. (Нижний Новгород, ННГУ им.~Н.И.~Лобачевского, 2018);
    \item XXVII научная конференция по радиофизике. (Нижний Новгород, ННГУ им.~Н.И.~Лобачевского, 2023);
    \item International Conference on Statistical Physics. (Греция, Корфу, 2017).
\end{itemize}


{\contribution} Все представленные в работе результаты были либо получены лично автором, либо при его непосредственном участии. Автор принимал прямое участие в постановке задач и анализе полученных результатов, а также в подготовке публикаций в научных журналах и докладов на тематических конференциях.

\ifnumequal{\value{bibliosel}}{0}
{%%% Встроенная реализация с загрузкой файла через движок bibtex8. (При желании, внутри можно использовать обычные ссылки, наподобие `\cite{vakbib1,vakbib2}`).
    {\publications} Основные результаты по теме диссертации изложены
    в~9~печатных изданиях,
    4 из которых изданы периодических научных журналах, индексируемых Web of~Science и Scopus,
    5 "--- в тезисах докладов.
    Зарегистрированы 2 программы для ЭВМ.
}%
{%%% Реализация пакетом biblatex через движок biber
    \begin{refsection}[bl-author, bl-registered]
        % Это refsection=1.
        % Процитированные здесь работы:
        %  * подсчитываются, для автоматического составления фразы "Основные результаты ..."
        %  * попадают в авторскую библиографию, при usefootcite==0 и стиле `\insertbiblioauthor` или `\insertbiblioauthorgrouped`
        %  * нумеруются там в зависимости от порядка команд `\printbibliography` в этом разделе.
        %  * при использовании `\insertbiblioauthorgrouped`, порядок команд `\printbibliography` в нем должен быть тем же (см. biblio/biblatex.tex)
        %
        % Невидимый библиографический список для подсчета количества публикаций:
        \printbibliography[heading=nobibheading, section=1, env=countauthorvak,          keyword=biblioauthorvak]%
        \printbibliography[heading=nobibheading, section=1, env=countauthorwos,          keyword=biblioauthorwos]%
        \printbibliography[heading=nobibheading, section=1, env=countauthorscopus,       keyword=biblioauthorscopus]%
        \printbibliography[heading=nobibheading, section=1, env=countauthorconf,         keyword=biblioauthorconf]%
        \printbibliography[heading=nobibheading, section=1, env=countauthorother,        keyword=biblioauthorother]%
        \printbibliography[heading=nobibheading, section=1, env=countregistered,         keyword=biblioregistered]%
        \printbibliography[heading=nobibheading, section=1, env=countauthorpatent,       keyword=biblioauthorpatent]%
        \printbibliography[heading=nobibheading, section=1, env=countauthorprogram,      keyword=biblioauthorprogram]%
        \printbibliography[heading=nobibheading, section=1, env=countauthor,             keyword=biblioauthor]%
        \printbibliography[heading=nobibheading, section=1, env=countauthorvakscopuswos, filter=vakscopuswos]%
        \printbibliography[heading=nobibheading, section=1, env=countauthorscopuswos,    filter=scopuswos]%
        %
        \nocite{*}%
        %
        {\publications} Основные результаты по теме диссертации изложены в~\arabic{citeauthor}~печатных изданиях,
        \arabic{citeauthorvak} из которых изданы в журналах, рекомендованных ВАК\sloppy%
        \ifnum \value{citeauthorscopuswos}>0%
            , \arabic{citeauthorscopuswos} "--- в~периодических научных журналах, индексируемых Web of~Science и Scopus\sloppy%
        \fi%
        \ifnum \value{citeauthorconf}>0%
            , \arabic{citeauthorconf} "--- в~тезисах докладов.
        \else%
            .
        \fi%
        \ifnum \value{citeregistered}=1%
            \ifnum \value{citeauthorpatent}=1%
                Зарегистрирован \arabic{citeauthorpatent} патент.
            \fi%
            \ifnum \value{citeauthorprogram}=1%
                Зарегистрирована \arabic{citeauthorprogram} программа для ЭВМ.
            \fi%
        \fi%
        \ifnum \value{citeregistered}>1%
            Зарегистрированы\ %
            \ifnum \value{citeauthorpatent}>0%
            \formbytotal{citeauthorpatent}{патент}{}{а}{}\sloppy%
            \ifnum \value{citeauthorprogram}=0 . \else \ и~\fi%
            \fi%
            \ifnum \value{citeauthorprogram}>0%
            \formbytotal{citeauthorprogram}{программ}{а}{ы}{} для ЭВМ.
            \fi%
        \fi%
        % К публикациям, в которых излагаются основные научные результаты диссертации на соискание ученой
        % степени, в рецензируемых изданиях приравниваются патенты на изобретения, патенты (свидетельства) на
        % полезную модель, патенты на промышленный образец, патенты на селекционные достижения, свидетельства
        % на программу для электронных вычислительных машин, базу данных, топологию интегральных микросхем,
        % зарегистрированные в установленном порядке.(в ред. Постановления Правительства РФ от 21.04.2016 N 335)
    \end{refsection}%
    \begin{refsection}[bl-author, bl-registered]
        % Это refsection=2.
        % Процитированные здесь работы:
        %  * попадают в авторскую библиографию, при usefootcite==0 и стиле `\insertbiblioauthorimportant`.
        %  * ни на что не влияют в противном случае
        %\nocite{vakbib2}%vak
        %\nocite{patbib1}%patent
        \nocite{progbib1}%program
        \nocite{bib1}%other
        \nocite{bib2}%other
        \nocite{confbib1}%conf
    \end{refsection}%
        %
        % Все, что вне этих двух refsection, это refsection=0,
        %  * для диссертации - это нормальные ссылки, попадающие в обычную библиографию
        %  * для автореферата:
        %     * при usefootcite==0, ссылка корректно сработает только для источника из `external.bib`. Для своих работ --- напечатает "[0]" (и даже Warning не вылезет).
        %     * при usefootcite==1, ссылка сработает нормально. В авторской библиографии будут только процитированные в refsection=0 работы.
}

% При использовании пакета \verb!biblatex! будут подсчитаны все работы, добавленные
% в файл \verb!biblio/author.bib!. Для правильного подсчета работ в~различных
% системах цитирования требуется использовать поля:
% \begin{itemize}
%         \item \texttt{authorvak} если публикация индексирована ВАК,
%         \item \texttt{authorscopus} если публикация индексирована Scopus,
%         \item \texttt{authorwos} если публикация индексирована Web of Science,
%         \item \texttt{authorconf} для докладов конференций,
%         \item \texttt{authorpatent} для патентов,
%         \item \texttt{authorprogram} для зарегистрированных программ для ЭВМ,
%         \item \texttt{authorother} для других публикаций.
% \end{itemize}
% Для подсчета используются счетчики:
% \begin{itemize}
%         \item \texttt{citeauthorvak} для работ, индексируемых ВАК,
%         \item \texttt{citeauthorscopus} для работ, индексируемых Scopus,
%         \item \texttt{citeauthorwos} для работ, индексируемых Web of Science,
%         \item \texttt{citeauthorvakscopuswos} для работ, индексируемых одной из трех баз,
%         \item \texttt{citeauthorscopuswos} для работ, индексируемых Scopus или Web of~Science,
%         \item \texttt{citeauthorconf} для докладов на конференциях,
%         \item \texttt{citeauthorother} для остальных работ,
%         \item \texttt{citeauthorpatent} для патентов,
%         \item \texttt{citeauthorprogram} для зарегистрированных программ для ЭВМ,
%         \item \texttt{citeauthor} для суммарного количества работ.
% \end{itemize}
% % Счетчик \texttt{citeexternal} используется для подсчета процитированных публикаций;
% % \texttt{citeregistered} "--- для подсчета суммарного количества патентов и программ для ЭВМ.

% Для добавления в список публикаций автора работ, которые не были процитированы в
% автореферате, требуется их~перечислить с использованием команды \verb!\nocite! в
% \verb!Synopsis/content.tex!. % Характеристика работы по структуре во введении и в автореферате не отличается (ГОСТ Р 7.0.11, пункты 5.3.1 и 9.2.1), потому её загружаем из одного и того же внешнего файла, предварительно задав форму выделения некоторым параметрам

%Диссертационная работа была выполнена при поддержке грантов \dots

%\underline{\textbf{Объем и структура работы.}} Диссертация состоит из~введения,
%четырех глав, заключения и~приложения. Полный объем диссертации
%\textbf{ХХХ}~страниц текста с~\textbf{ХХ}~рисунками и~5~таблицами. Список
%литературы содержит \textbf{ХХX}~наименование.

\pdfbookmark{Содержание работы}{description}                          % Закладка pdf
\section*{Содержание работы}
Во \underline{\textbf{введении}} обосновывается актуальность
исследований, проводимых в~рамках данной диссертационной работы,
формулируется цель, ставятся задачи работы, излагается научная новизна
и практическая значимость представляемой работы. 

\underline{\textbf{Первая глава}} посвящена обзору научной литературы 
по теории антагонистических игр, теории случайных блужданий, теории проведения социальных экспериментов
с применением мобильных технологий. Дополнительно приводится описание 
разработанной игры Random Walk Game, на основе которой строится дальнейшее повествование.

В \underline{\textbf{разделе 1.1.1}} рассмотрена задача о разорении игрока.
Первые упоминания о задаче разорения игрока возникли в переписке
Блеза Паскаля и Пьера Ферма в 1656 году при рассмотрении игры тремя костями
между двумя игроками. Пьер де Каркави переформулировал задачу в письме Христиану Гюйгенсу, 
и поставил вопрос о нахождении вероятности победы первого и второго игрока.
Христиан Гюйгенс обновил формулировку задачи в своей работе и приблизил ее к классической:
у игроков есть конечно количество монет, на каждом ходу подбрасывается ассиметричная монета
и выпадение аверса или реверса определяет какой игрок какому отдает одну монета.
Игрок, оставшийся без монет считается проигравшим. В дополнении к вопросу о вероятностях выигрыша 
ставится вопрос среднего времени игры.

Поиск решений с использованием подхода теории случайных блужданий приводит к нахождению выражений в замкнутой форме 
для оценки как вероятности победы каждого из игроков, так и среднего количества ходов в игре в зависимости от начальных капиталов игроков.
В предположении симметричности монеты и одинакового капитала игроков среднее время игры составляет $B^2$, где $B$ -- капитал каждого из игроков.

Продолжая анализ случайных блужданий в \underline{\textbf{разделе 1.1.2}} рассматривается
обобщении задачи о разорении игрока на большое количество измерений.
Один из таких подходов обобщения был предложен израильскими математиками рассматривавшими игру 
с учетом нескольких валют, имеющихся у игроков, при этом завершение игры определяется
когда у одного из игроков заканчиваются монеты любой из валют. 
Решение задачи о поиске среднего времени игры было найдено с применением дискретного преобразования Фурье
и свойств спектра для матрицы Теплица. 

В \underline{\textbf{разделе 1.1.3}} проводится обзор научной литературы, демонстрирующей
междисциплинарную природу возникновения процессов случайных блужданий. Приводятся
такие работы как задача Карла Пирсона, возникшая в биологии, но как оказалось уже решенная ранее
Лордом Рэлеем при анализе колебаний в теории звука. Исследование свойств конфигураций гибких молекул 
в химии в работах Куна привело к задаче свободно сочлененной цепи со звеньями фиксированной длины,
но случайно ориентацией. Возникшая задача представляла собой обобщение задачи Пирсона на 
трехмерный случай. В другой работе Н.~Темперлей заметил, что физические явления, связанные с решетками
могут быть решены в терминах случайных блужданий на решетках с ограничениями.

Возникновение случайных блужданий в физике при рассмотрении модели ферромагнетизма на плоскости привело к появлению 
новой задачи анализа случайных блужданий без самопересечений. Один из важных классов случайных процессов
был выделен Андреем Марковым. Класс Марковских процессов обладает характеристикой независимости
будущих состояний от прошлых состояний при определенном настоящем состоянии. В конце раздела приводятся 
объекты, на которых можно осуществлять случайные блуждания.

В следующем \underline{\textbf{разделе 1.1.4}} приводится обобщение задачи о разорении игрока 
на случай произвольной многошаговой игры в условиях неопределенности. Работы Нэша о
некооперативных играх с нулевой суммой предлагают общие свойства таких игр и наличие 
равновесия в стратегиях игроков. Исследования Беллмана показали приблизительное решение 
для оптимальных стратегий игроков в произвольных многошаговых играх. В своих дальнейших работах 
Джон Милнор и Лоойд Шепли нашли точное решение игры. Дополнительное обобщение игры
было предложено И. В. Романовским на случай влияния случайной компоненты на выигрыш игроков 
на каждом ходе. 

\underline{\textbf{Раздел 1.1.5}} представляет анализ социальных экспериментов 
с применением новых технологий, таких как мобильные приложения и интернет-технологии.
В разделе характеризуется развитие сотовой связи и электроники, позволившего связать
людей независимо от их георасположения на планете. Анализ литературы продемонстрировал,
что большая часть экспериментов применяющих мобильные приложения связаны со здравоохранением.
Однако применения были найдены и в других сферах, таких как экономика, мониторинг эффективности программ обучения,
социальных явлениях и других. 

Сбор данных возможен как с датчиков устройства, так и информация о действиях пользователя,
а также записи в дневнике о полученном опыте участника эксперимента. Дополнительная информация
возникает при столкновении в игре человека с компьютерной системой или с другими игроками.
Таким образом применение мобильных приложений существенно расширило возможности  
по автоматическому и полуавтоматическому сбору данных и упростило методы проведения полевых экспериментов.

В \underline{\textbf{разделе 1.2}} предлагается новый игровой процесс между двумя игроками, управляющих
совместным выбором стохастическим движением фишки на дискретной решетке. Предложенная игра
является частным случаем игр, рассматриваемых И.~В.~Романовским, учитывающим свойство ограниченности пространства
и случайную компоненту в выборе игроков. Игроки обладают противоположными целями по оптимизации 
времени игры, то есть числа ходов: первый игрок старается максимизировать время игры, тогда как второй -- минимизировать его.
По достижении границы поля количество ходов, совершенных игроками, является временем игры и соответствует результату игры.


% картинку можно добавить так:
% \begin{figure}[ht]
%     \centerfloat{
%         \hfill
%         \subcaptionbox{\LaTeX}{%
%             \includegraphics[scale=0.27]{latex}}
%         \hfill
%         \subcaptionbox{Knuth}{%
%             \includegraphics[width=0.25\linewidth]{knuth1}}
%         \hfill
%     }
%     \caption{Подпись к картинке.}\label{fig:latex}
% \end{figure}

% Формулы в строку без номера добавляются так:
% \[
%     \lambda_{T_s} = K_x\frac{d{x}}{d{T_s}}, \qquad
%     \lambda_{q_s} = K_x\frac{d{x}}{d{q_s}},
% \]

\underline{\textbf{Вторая глава}} описывает методы, подходы и алгоритмы, применяемые для анализа игрового процесса.
В ходе описания приводится подробное построение стохастической модели игры и способы получить основные статистические свойства
процесса. В качестве подходов рассматриваются три основных инструмента: моделирование эволюции вероятности,
расчет фундаментальной матрицы и численное моделирование методом Монте-Карло. 

Модель игры представляет собой марковскую цепь с поглощающими состояниями, при этом вероятности перехода
определяются совместным распределением стратегий двух игроков.
Применяя теорию поглощающих Марковских цепей было получено решение задачи о вычислении среднего времени игры при заданных
вероятностях выборов игроками для каждого состояния. 

Однако, оценка распределения времен не была найдена в замкнутой форме.
В связи с чем был предложен подход к моделированию эволюции вероятности найти фишку в некотором состоянии на решетке.
Используя модифицированную матрицу вероятностей переходов были получены оценки для распределения времен окончания игры,
а также пространственные распределения на каждом шаге и свойства четности времени игры.

Дополнительным подходом, расширяющим информацию не только о статистических свойствах игры, но также и об 
индивидуальных траекториях, является численное моделирование. Применяя алгоритм Монте-Карло для симуляции 
траекторий можно получить дополнительную информацию о структуре отдельных траекторий. Для визуализации были
предложены различные способы, такие как визуализация всех ходов траектории на плоскости, визуализация ходов последовательными блоками
и анимация движения фишки на поле. 

Хотя анализ теоретических аспектов игры возможен благодаря рассмотренным методам,
выявление особенностей, возникающих у людей при игре в Random Walk Game, требует 
проведения натурного эксперимента с участием реальных игроков. В \underline{\textbf{разделе 2.1.6}}
описываются несколько различных организационных мероприятий, в рамках которых игрокам 
было предложено соревноваться друг с другом в игре. В результате было проведено около полутора тысяч
матчей в течение 250 часов игр.

Проведение такого полномасштабного эксперимента стало возможно с использованием 
разработанного мобильного приложения (Рис.~\cref{fig:screenshot_game_field_ref}) и примененного подхода к автоматическому сбору 
данных о траекториях и выборах игроков в режиме онлайн посредством сети Интернет.
Объединение игроков в онлайн приложении позволило привлечь участников по всему миру в условиях ограничений,
связанных с пандемией и успешно реализовать эксперимент. Для проектирования приложения был
выбран подход MVVM и кроссплатформенная технология Xamarin на базе языка C$\mathsf{\#}$.

\begin{figure}[ht]
    \centerfloat{
        \includegraphics[scale=0.15]{screenshot_game_field}
    }
    \caption{
        Скриншот приложения Random Walk Game. Строка заголовка состоит из текущего количества ходов, 
        целей игроков и количества ходов в самой длинной игре. Игрок видит игровое поле, положение фишки и траекторию фишки. 
        Внизу экрана показана управляющая матрица $2$ на $2$, которая определяет результат совместного выбора стратегий игроками A и B. 
        Строки представляют собой возможный выбор стратегий для игрока A, а столбцы -- для игрока B. Результирующее направление движения 
        определяется стрелкой в ячейке, расположенной в соответствующих строке и столбце.
    }\label{fig:screenshot_game_field_ref}
\end{figure}

\underline{\textbf{Третья глава}} раскрывает результаты численного анализа времен игры для
различных случаев с применением различных численных подходов, рассмотренных во второй главе.
В главе также проводится анализ стратегий игроков и статистических свойств распределений 
для полученных экспериментальных игр. Дополнительно вводится гипотеза об оптимальных временах игры
для случаев игры против стратегии равновероятного выбора.

В \underline{\textbf{разделе 3.1}} рассматривается среднее время игры для различных случаев игрового взаимодействия:
чистое случайное блуждание (выборы для обоих игроков равновероятно распределены и не зависят от положения фишки, BvB),
один игрок выбирает произвольную стратегию против стратегии равновероятного выбора (PvE), 
оба игрока выбирают произвольную стратегию (PvP). Случай BvB вырождается в стандартную задачу случайного блуждания,
на квадратной решетке, для которой ранее были получены результаты в виде двойной суммы оценки среднего времени игры.
Случай PvE представляет собой задачу глобальной оптимизации с ограничениями, решение которой было найдено
для размерностей поля $5$ и $7$ с применением техник символьных вычислений в математическом пакете Wolfram Mathematica.

Значения среднего времени игры для малых размеров поля и особенности полученных стратегий, привели к появлению гипотезы
об оптимальном среднем времени игры для случая PvE. Используя сведение к одномерной Марковской цепи была показана
достижимость предполагаемых оптимальных средних времен, которые при игре за центр представляются в виде: $\boldsymbol{\mathsf{t_n^{PvE A}}} = (n-2)^2$, а в случае
игры за границу в виде: $\boldsymbol{\mathsf{t_n^{PvE B}}} = \frac{(n-1)^2}{4}$. 
Предложены различные стратегии, достигающие данных оценок \cite{confbib1}. Найденные оценки зависимости среднего времени игры 
от размера поля для трех случаев представляют собой квадратичную функцию с различными коэффициентами.

Сравнение результатов модели и эксперимента показывает высокую точность соответствия среднего времени игры между ними для всех случаев игры.
Однако, рассматривая случай игры двух игроков были обнаружены сверхдлинные игры, наличие которых было крайне маловероятно,
учитывая найденное распределение времен игры при общей стратегии всех участников. 
Анализ таких игр показал отклонение статистики среднего времени игры от модели столкновения индивидуальных стратегий.
Такое несоответствие связано с отсутствием свойства марковости процесса игры в этих отдельных случаях,
что в свою очередь может быть связано с возникновением <<синхронизации>> принятия решений игроками
в условиях накопленной усталости.

Дальнейший анализ проводился для четности времени игры, в котором была выявлена неравномерность 
вероятности закончить игру на четном и нечетном ходе. При этом соотношение между этими двумя величинами
различается в зависимости от случая игры. 

В заключительной части главы было проведено исследование распределения вероятности обнаружения фишки в некотором состоянии и 
частот выборов игроков в зависимости от состояния на решетке, характеризующих обобщенные стратегии группы участников эксперимента.
Была показана схожесть паттернов стратегий игроков с оптимальными стратегиями для всех случаев игры.



% Можно сослаться на свои работы в автореферате. Для этого в файле
% \verb!Synopsis/setup.tex! необходимо присвоить положительное значение
% счётчику \verb!\setcounter{usefootcite}{1}!. В таком случае ссылки на
% работы других авторов будут подстрочными.
% Изложенные в третьей главе результаты опубликованы в~\cite{vakbib1, vakbib2}.
% Использование подстрочных ссылок внутри таблиц может вызывать проблемы.

\FloatBarrier
\pdfbookmark{Заключение}{conclusion}                                  % Закладка pdf
В \underline{\textbf{заключении}} приведены основные результаты работы, которые заключаются в следующем:
%% Согласно ГОСТ Р 7.0.11-2011:
%% 5.3.3 В заключении диссертации излагают итоги выполненного исследования, рекомендации, перспективы дальнейшей разработки темы.
%% 9.2.3 В заключении автореферата диссертации излагают итоги данного исследования, рекомендации и перспективы дальнейшей разработки темы.
\begin{enumerate}
  \item Построена стохастическая модель, описывающая предложенный игровой конфликт двух игроков, управляющих блужданием фишки на конечной квадратной решетке. 
  \item Разработаны методы для расчета статистических характеристик игрового процесса при фиксированных заданных стратегиях игроков,
        таких как среднее время игры, распределение времен игры, распределение вероятностей наблюдения фишки в состояниях конечной решетки. 
  \item Выдвинута гипотеза об оптимальном среднем времени для двух случаев игры против стратегии случайного равновероятного выбора,
        и установлена оптимальность данных значений для малых размеров игрового поля, а также найден класс стратегий, достигающих предполагаемой оптимальной оценки.
  \item Разработано мобильное приложение Random Walk Game, реализующее игровую механику посредством сети интернет с использованием
        созданного веб-сервера по обработке и хранению результатов игр участников. Дополнительно разработан веб-сайт для
        отображения статистической информации по результатам игр участников в режиме реального времени.
        Приложение опубликовано в открытом доступе на двух маркетплейсах для платформ Android и iOS.
  \item Проведен масштабный эксперимент с применением мобильных и интернет-технологий, привлекший более 100 участников
        и позволивший собрать более 1500 игр Random Walk Game. 
  \item Подтверждено соответствие эксперимента предложенной модели для трех случаев игры на основе сравнения
        распределений и соответствующих средних времен игры. 
  \item Установлены возникающие особенности синхронизации игроков при длительных играх, связанные со снижением концентрации игроков.
        Длительные игры утомляют игроков, что снижает способность человека генерировать чисто случайную последовательность выборов.
  \item Выявлены отличия стратегий игроков от предполагаемых оптимальных стратегий
        на основе сравнения статистических свойств траекторий и стратегий участников для случаев игры против стратегии равновероятного случайного выбора.
\end{enumerate}


\pdfbookmark{Литература}{bibliography}                                % Закладка pdf
% При использовании пакета \verb!biblatex! список публикаций автора по теме
% диссертации формируется в разделе <<\publications>>\ файла
% \verb!common/characteristic.tex!  при помощи команды \verb!\nocite!

\ifdefmacro{\microtypesetup}{\microtypesetup{protrusion=false}}{} % не рекомендуется применять пакет микротипографики к автоматически генерируемому списку литературы
\urlstyle{rm}                               % ссылки URL обычным шрифтом
\ifnumequal{\value{bibliosel}}{0}{% Встроенная реализация с загрузкой файла через движок bibtex8
    \renewcommand{\bibname}{\large \bibtitleauthor}
    \nocite{*}
    \insertbiblioauthor           % Подключаем Bib-базы
    %\insertbiblioexternal   % !!! bibtex не умеет работать с несколькими библиографиями !!!
}{% Реализация пакетом biblatex через движок biber
    % Цитирования.
    %  * Порядок перечисления определяет порядок в библиографии (только внутри подраздела, если `\insertbiblioauthorgrouped`).
    %  * Если не соблюдать порядок "как для \printbibliography", нумерация в `\insertbiblioauthor` будет кривой.
    %  * Если цитировать каждый источник отдельной командой --- найти некоторые ошибки будет проще.
    %
    %% authorvak
    %\nocite{vakbib1}%
    %\nocite{vakbib2}%
    %
    %% authorwos
    %\nocite{wosbib1}%
    %
    %% authorscopus
    %\nocite{scbib1}%
    %
    %% authorpathent
    %\nocite{patbib1}%
    %
    %% authorprogram
    %\nocite{progbib1}%
    %
    %% authorconf
    \nocite{confbib1}%
    %\nocite{confbib2}%
    %
    %% authorother
    \nocite{bib1}%
    \nocite{bib2}%

    \ifnumgreater{\value{usefootcite}}{0}{
        \begin{refcontext}[labelprefix={}]
            \ifnum \value{bibgrouped}>0
                \insertbiblioauthorgrouped    % Вывод всех работ автора, сгруппированных по источникам
            \else
                \insertbiblioauthor      % Вывод всех работ автора
            \fi
        \end{refcontext}
    }{
        \ifnum \totvalue{citeexternal}>0
            \begin{refcontext}[labelprefix=A]
                \ifnum \value{bibgrouped}>0
                    \insertbiblioauthorgrouped    % Вывод всех работ автора, сгруппированных по источникам
                \else
                    \insertbiblioauthor      % Вывод всех работ автора
                \fi
            \end{refcontext}
        \else
            \ifnum \value{bibgrouped}>0
                \insertbiblioauthorgrouped    % Вывод всех работ автора, сгруппированных по источникам
            \else
                \insertbiblioauthor      % Вывод всех работ автора
            \fi
        \fi
        %  \insertbiblioauthorimportant  % Вывод наиболее значимых работ автора (определяется в файле characteristic во второй section)
        \begin{refcontext}[labelprefix={}]
            \insertbiblioexternal            % Вывод списка литературы, на которую ссылались в тексте автореферата
        \end{refcontext}
        % Невидимый библиографический список для подсчёта количества внешних публикаций
        % Используется, чтобы убрать приставку "А" у работ автора, если в автореферате нет
        % цитирований внешних источников.
        \printbibliography[heading=nobibheading, section=0, env=countexternal, keyword=biblioexternal, resetnumbers=true]%
    }
}
\ifdefmacro{\microtypesetup}{\microtypesetup{protrusion=true}}{}
\urlstyle{tt}                               % возвращаем установки шрифта ссылок URL
