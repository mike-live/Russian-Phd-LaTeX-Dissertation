
{\actuality} 

Исследование флуктуаций, шумов, случайных процессов и полей в сосредоточенных и распределенных стохастических системах составляет предметную, и проблема случайных блужданий является одной из наиболее значимых задач. Фундаментальные результаты представляют интерес со стороны широкого спектра прикладных задач, таких как: моделирование аномальной диффузии при магнитно-резонансной томографии, моделирование диффузии потоков в плазме, моделирование долговременных изменений климата, задачи турбулентной диффузии в гидродинамике, описание процесса передачи электрического тока по линиям электропередач, моделирование рассеяния радиоволн в межзвездной плазме, описание распространения волн в случайной среде, динамика распространения волн при землетрясениях и многих других. Теория случайных блужданий является ярким примером тесного взаимодействия методов математики и статистической радиофизики с различными науками, такими как химия, физика, экономика, биология и другими.

Существенный вклад в развитие методов статистической радиофизики применительно к задачам случайных блужданий внесли ведущие отечественные и зарубежные ученые (Metzler R., Klafter J., Chechkin A.V., Spagnolo B., Dybiec B., Gudowska—Nowak E., Sokolov I.M., Barkai E., Hänggi P., Дубков А.А., Панкратов А.Л., Агудов Н.В., Руденко О.В., Ряшко Л.Б., Башкирцева И.А., Учайкин В.В., Забурдаев В.Ю., Романовский И.В., Гнеденко Б.В. и другие).

Интерес к дальнейшему развитию теории случайных блужданий стимулируется множеством приложений, как уже традиционными, например: оценка качества работы поисковой системы в сети интернет, описание физики полимеров, описание движения цен активов на фондовом рынке, -- так и новыми, возникающими в связи с необходимостью учета дискретности системы, например: анализ эффектов термической обработки при производстве полупроводников в условиях малых пространственных масштабов элементов, сегментация изображений в компьютерном зрении, описание динамики движения сообществ (популяций) бактерий и животных в процессе поиска пищи, описание мелких движений глаза и многими другими. Несмотря на высокую степень сформированности теории, вопросы, возникающие в новых приложениях, открывают возможности для дальнейших исследований. 

Одной из важных прикладных задач теории случайных блужданий является исследование стратегий поиска, в первую очередь, применительно к природным системам. Существование и выживание многих биологических видов возможно благодаря таксису, то есть способности живых организмов двигаться в направлении максимальной концентрации необходимых для их жизнедеятельности ресурсов (пищи, воды, химических агентов) и возрастающей оптимальности условий существования (нахождение области с наиболее приемлемой комбинацией температуры, влажности, освещенности и другими факторами). Очевидно, что чем лучше стратегия таксиса, используемая организмом, тем выше его индивидуальная выживаемость и эволюционные преимущества его вида в целом.

Примером таксиса является хемотаксис -- движение бактерии в направлении максимальной концентрации химического аттрактанта. Линейные размеры бактерии слишком малы, чтобы оценивать градиент как разницу концентраций, измеренных в разных частях клетки. Вместо этого бактерия использует механизм памяти, <<интегрирующий>> измеренную концентрацию во времени, что позволяет бактерии, используя стратегию случайных поворотов, частота которых зависит от предыстории изменения концентрации, двигаться в направлении источника аттрактанта. В рамках радиофизического подхода такая система описывается моделью случайных блужданий, в которой частота зависит от свертки функции отклика бактерии с функцией концентрации химического вещества. Неисследованным остается вопрос анализа стратегий движения, управляемых двумя чередующимися углами, имитирующих движение таких видов бактерий как V.~alginolyticus. В настоящей диссертационной работе предлагается расширение существующей теории для учета паттерна, состоящего из нескольких различных событий поворота.

Исследование закономерностей управления переключениями между стадиями движения является самостоятельной задачей. Экспериментальные данные демонстрируют как наличие экспоненциальных распределений длительностей, так и возникновение степенных асимптотик в распределении на некотором участке. До настоящего времени описание базовых механизмов, стоящих за подобными закономерностями, в рамках простых математических моделей отсутствовало.

Получение локальных сигналов, указывающих на расположение источника, не всегда возможно в макроскопических масштабах, потому что динамика среды, в которой движется организм (воздух, жидкость), привносит существенный фактор помех и случайных флуктуаций концентраций вещества. Таким образом, животные, ощущающие запахи в воздухе или воде, обнаруживают их только эпизодически, как <<пятна>> концентрации, постоянно изменяемые ветром или турбулентным течением. Организм -- природный или искусственный (как, например, автономный робот) -- нуждается в оптимальной стратегии движения, использующей спорадические сигналы и частичную информацию.

Такие стратегии получили название <<инфотаксиса>>. Процесс поиска можно рассматривать как получение информации об исходном местоположении. Информация в данном процессе играет роль, аналогичную концентрации в хемотаксисе. Оптимальные стратегии инфотаксиса локально максимизируют ожидаемую скорость получения информации. Существующие алгоритмы инфотаксиса приводят к траекториям, которые характеризуются <<зигзагообразной>> структурой, наблюдаемой при полете насекомых. Использование систем из множества агентов имеет больше преимуществ, чем одиночный робот при локализации источника за счет возможности уменьшения ожидаемого времени поиска и уменьшения вероятности попадания в локальные экстремумы.

Процессы таксиса могут быть обусловлены влиянием окружающей среды, имеющей как случайную компоненту, так и целенаправленную компоненту, в качестве которой может выступать другой вид организмов, конфликтующий за ресурсы и условия существования. Открытым вопросом, исследуемым в настоящей диссертационной работе, является анализ влияния случайной компоненты в стратегиях хемотаксиса бактерий, в частности, построение простой модели, описывающей механизм возникновения блужданий Леви у бактерий. Вопросы влияния второй компоненты, связанной с возникающим конфликтом противоборствующих сторон на блуждания агента, также остаются неисследованными ввиду сложности проведения натурного эксперимента. В рамках предложенной модели игровых случайных блужданий в настоящей диссертационной работе решается проблема проведения такого эксперимента и раскрываются особенности стратегий как в случае влияния случайной компоненты, так и в случае наличия конфликтующей стороны. 

Игровые взаимодействия, в которых каждый участник руководствуется принципами максимизации своего личного выигрыша или целого коллектива в некоторой игре, существенно отличны от обычных физических взаимодействий. Дополненные механизмами обучения, оценки и адаптации игровые взаимодействия определяют принципиально новый тип динамики.

В качестве модельного процесса в настоящей диссертационной работе предложена игра, в которой оппоненты управляют перемещением агента на квадратной решетке, делая независимый выбор одной из двух возможных стратегий на каждом шаге игры. Информация о возможных перемещениях агента открыта для обоих игроков и организована в виде матрицы. Целью первого игрока является максимально долгое удержание агента внутри ограниченной области, а второго -- максимально быстрое достижение им поглощающей границы. Результатом игры является время поглощения. 

Игры такого типа были предложены в ряде работ 1950-1960х годов. Однако в силу вычислительной сложности задачи, количественные результаты были получены только для очень простых моделей с тремя-пятью состояниями, что не позволяет делать заключения об асимптотических характеристиках процесса и использовать для их анализа статистический подход. Существует тесная связь между <<играми на выживание>> и процессами случайных блужданий, например, классическая задача о разорении игрока допускает прямую интерпретацию как процесс случайного блуждания на конечном интервале. 

С другой стороны, проблема случайных блужданий в ограниченной области и такие вопросы, как оценка среднего времени достижения границы (или любого другого определенного региона), в настоящее время переживает очередную волну интереса, вызванную перспективами приложения этого подхода к проблемам молекулярной биологии, химической кинетики и экологии. 

Идея данной работы состоит в том, что игровые блуждания в конечных пространственных областях, ограниченных поглощающими границами, могут быть исследованы и квантифицированы с использованием методологии теории случайных блужданий и теории игр. Данная диссертационная работа содержит результаты проведения масштабного натурного эксперимента с участием реальных игроков. Данная задача лежит на стыке теории случайных блужданий и теории игр и имеет значение в контексте прикладных вопросов биофизики и социологии.

На основе сказанного можно сделать вывод о том, что актуальность научного направления и тема настоящей диссертационной работы сочетается с научными интересами широкого круга специалистов в мировой науке и является востребованной и важной для исследований в современной радиофизике.

% В соответствии с паспортом специальности 01.04.03 <<Радиофизика>>, данная диссертация
% относится к области исследований: <<4. Исследование флуктуаций, шумов, случайных процессов и полей в сосредоточенных и распределенных стохастических системах (статистическая радиофизика). Создание новых методов анализа и статистической обработки сигналов в условиях помех. Разработка статистических основ передачи информации. Исследование нелинейной динамики, пространственно-временного хаоса и самоорганизации в неравновесных физических, биологических, химических и экономических системах.>>, <<4. Реализация эффективных численных методов и алгоритмов в виде
% комплексов проблемно-ориентированных программ для проведения
% вычислительного эксперимента>>.

\ifsynopsis
% Этот абзац появляется только в~автореферате.
% Для формирования блоков, которые будут обрабатываться только в~автореферате,
% заведена проверка условия \verb!\!\verb!ifsynopsis!.
% Значение условия задается в~основном файле документа (\verb!synopsis.tex! для
% автореферата).
\else
%Этот абзац появляется только в~диссертации.
%Через проверку условия \verb!\!\verb!ifsynopsis!, задаваемого в~основном файле
%документа (\verb!dissertation.tex! для диссертации), можно сделать новую
%команду, обеспечивающую появление цитаты в~диссертации, но~не~в~автореферате.
\fi

% {\progress}
% Этот раздел должен быть отдельным структурным элементом по
% ГОСТ, но он, как правило, включается в описание актуальности
% темы. Нужен он отдельным структурынм элемементом или нет ---
% смотрите другие диссертации вашего совета, скорее всего не нужен.

{\aim} данной работы является развитие теории случайных стратегий поиска и механизмов направленного случайного блуждания, а также развитие теории случайных стратегий игровых блужданий в ограниченной двухмерной области.

Для~достижения поставленной цели необходимо было решить следующие {\tasks} диссертационного исследования:
\begin{enumerate}[beginpenalty=10000] % https://tex.stackexchange.com/a/476052/104425
    \item Получить аналитическую форму средней скорости в математической модели для колонии бактерий в случае паттерна движения с двумя чередующимися углами и подтвердить ее в численном эксперименте при малом химическом градиенте.
    \item Построить и проанализировать стохастическую модель влияния генетического шума на частоту переключения вращения моторов у бактерий. Получить способ оценки распределения длительностей для модели и подобрать параметры для генерации двух режимов: степенное распределение и экспоненциальное распределение длительностей.
    \item Разработать стохастическую модель блужданий на плоскости, управляемых игровым конфликтом, получить статистические характеристики игрового процесса, провести анализ стратегий и найти оптимальные стратегии в предложенной игре. 
    \item Провести серию экспериментов с участием реальных игроков, получить статистически значимый массив данных и провести анализ соответствия эксперимента и модели.
\end{enumerate}


{\novelty}
В работе получены следующие новые научные результаты:
\begin{enumerate}[beginpenalty=10000] % https://tex.stackexchange.com/a/476052/104425
  \item Получена аналитическая формула для расчета средней скорости колонии бактерий в случае паттерна движения с двумя чередующимися углами. Впервые предложен подход для нахождения средней скорости колонии бактерий при стратегии движения с произвольным конечным количеством чередующихся поворотов методом решения системы линейных уравнений.
  \item Впервые предложена математическая модель генной сети, генерирующей степенное распределение длительностей вращения жгутиковых моторов за счет генетического шума. Продемонстрировано применение подхода нахождения распределения длительностей к предложенной математической модели.
  \item Численно исследованы параметры модели и найдены различные комбинации параметров, демонстрирующие смену режима с экспоненциального распределения длительностей на степенное распределение длительностей при переключении направления вращения моторов. Впервые продемонстрировано, что степенные распределения возникают при времени релаксации, существенно превышающем время переключения между направлениями вращения моторов.
  \item Впервые предложен игровой конфликт двух игроков, управляющих блужданием фишки на плоскости, реализованный в виде программного обеспечения -- мобильного приложения. Новизна подхода заключается в применении интернет-технологий для реализации игры, одновременно учитывающей возможность создания как  игрового взаимодействия между игроками, так и процесса случайного блуждания.
  \item Впервые разработана стохастическая модель предложенной игровой динамики, позволяющая получить характеристики игрового процесса и воспроизвести результаты игр реальных игроков, собранные в масштабном эксперименте. Новизна подхода состоит в возможности исследования модельного процесса случайного блуждания, позволяющего воспроизвести экспериментально полученные характеристики, а также провести точное сравнение модели с натурным экспериментом.
  \item Впервые разработаны и реализованы численные методы для расчета статистических характеристик игрового процесса при фиксированных заданных стратегиях игроков, таких как среднее время игры, распределение времен игры, распределение вероятностей наблюдения фишки в состояниях конечной решетки. 
  \item Впервые найдены оптимальные средние времена для трех случаев игры, предложены классы оптимальных стратегий и визуализированы конкретные стратегии. Предложен подход для нахождения оптимальных стратегий при произвольной стратегии оппонента.
  \item Впервые построена модель когнитивного возраста на основе подходов машинного обучения для оценки когнитивного статуса индивидуума.
\end{enumerate}

{\influence} состоит в рассмотрении вопросов влияния случайной и противодействующей компонент на стратегию случайных блужданий в рамках модели хемотаксиса и предложенной игровой модели. Исследование таксиса сопряжено с поиском и анализом стратегий наиболее применимых в условиях возникающих спорадических сигналов и частичной информации. Результаты работы могут иметь применимость как в конкретных сферах (редактирование генома бактерий для достижения наиболее эффективной таргетной доставки лекарств в организме), так и в сферах управления искусственными мультиагентными системами при решении задачи поиска некоторого источника. Рассмотрение предложенной модели игровых блужданий является отправной точкой для теоретических исследований конфликта между мультиагентными системами и между агентами и средой.

{\methods} 
В работе используются методы математического моделирования, статистической радиофизики, теории игр, теории вероятностей, теории марковских цепей, теории случайных процессов, теории случайных блужданий и численного моделирования. Дополнительно используется подход к проведению масштабного натурного эксперимента с применением интернет-технологий и мобильных приложений.


{\defpositions}
\begin{enumerate}[beginpenalty=10000] % https://tex.stackexchange.com/a/476052/104425
    \item Математическое ожидание скорости смещения в модели колонии бактерий в условиях малого постоянного градиента химического вещества для паттерна движения, характеризующегося двумя чередующимися углами, обратно пропорционально величине базовой частоты переключения состояний и третьей степени коэффициента вращательной диффузии. Учет вращательной диффузии в системе уменьшает математическое ожидание скорости смещения, а также приводит к появлению точки максимума скорости, наблюдаемой при совпадении углов.
    \item Смена экспоненциальной асимптотики распределения длительностей нахождения в одном из двух состояний на степенную возникает в простой модели химической кинетики за счет дробового шума при существенном превышении времени релаксации уровня скрытой переменной над временем переключения между парой состояний. 
    \item Предложенная игра двух оппонентов сводится к рекурсивной игре, для которой оптимальное среднее время игры определяется как центральный элемент вектора неподвижной точки преобразования цен, вычисляемого как набор цен матричных игр, рекурсивно определяемых в каждом узле игровой решетки в соответствии с правилами игры. Классу оптимальных стратегий принадлежат: (1) стратегия, ограничивающая выход за границу только в угловых узлах на главной диагонали в случае игры за центр против стратегии случайного равновероятного выбора; (2) стратегия, ограничивающая блуждание игрока за границу против стратегии случайного равновероятного выбора на вертикальном и горизонтальном отрезках, проходящих через центр.
    \item Экспериментально полученные популяционные стратегии в случае игры против случайного выбора отклоняются от оптимальных стратегий: (1) при игре за центр за счет ненулевой частоты окончания игр в узлах, отличных от угловых; (2) при игре за границу за счет попыток исследования игроками всей двумерной области поля. 
\end{enumerate}

{\reliability} полученных результатов, научных положений и выводов, полученных в диссертации, обеспечивается корректным обоснованием постановок задач, точной формулировкой критериев, подтверждается результатами вычислительных экспериментов по использованию предложенных в диссертации методов и алгоритмов, сравнением полученных результатов с проведенными ранее исследованиями и перекрестной проверкой с применением различных методов. Результаты находятся в соответствии с результатами, полученными другими авторами.

{\probation}
Основные результаты диссертационного исследования были представлены на следующих научных конференциях и фестивалях в 2017-2023 гг. в форме секционных и стендовых докладов:
\begin{itemize}
    \item XXVI научная конференция по радиофизике, посвященная 120-летию со дня рождения М.Т. Греховой. (Нижний Новгород, ННГУ им.~Н.И.~Лобачевского, 2022);
    \item Всероссийский фестиваль молодежных инноваций Иннофест. (Нижний Новгород, ННГУ им.~Н.И.~Лобачевского, 2020);
    \item 74-я всероссийская с международным участием школа-конференция молодых ученых, посвященная памяти проф. А.П. Веселова <<Биосистемы: организация, поведение, управление>>. (Нижний Новгород, ННГУ им.~Н.И.~Лобачевского, 2021);
    \item 3rd International Conference Volga Neuroscience Meeting 2021. (Нижний Новгород, Отель <<Чайка>>, 2021);
    \item Нелинейные дни в Саратове для молодых: сборник научных трудов. (Саратов, СГУ им.~Н.Г.~Чернышевского, 2023)
    \item 71-я всероссийская с международным участием школа-конференция молодых ученых <<Биосистемы: организация, поведение, управление>>. (Нижний Новгород, ННГУ им.~Н.И.~Лобачевского, 2018);
    \item XXVII научная конференция по радиофизике. (Нижний Новгород, ННГУ им.~Н.И.~Лобачевского, 2023);
    \item International Conference on Statistical Physics. (Греция, Корфу, 2017).
\end{itemize}


{\contribution} Все представленные в работе результаты были получены либо лично автором, либо при его непосредственном участии. Автор принимал прямое участие в постановке задач и анализе полученных результатов, а также в подготовке публикаций в научных журналах и докладов на тематических конференциях.

\ifnumequal{\value{bibliosel}}{1}
{%%% Встроенная реализация с загрузкой файла через движок bibtex8. (При желании, внутри можно использовать обычные ссылки, наподобие `\cite{vakbib1,vakbib2}`).
    {\publications} Основные результаты по теме диссертации изложены
    в~9~печатных изданиях,
    4 из которых опубликованы в периодических научных журналах, индексируемых Web of~Science и Scopus,
    5 "--- в тезисах докладов.
    Зарегистрированы 2 программы для ЭВМ.
}%
{%%% Реализация пакетом biblatex через движок biber
    \begin{refsection}[bl-author, bl-registered]
        % Это refsection=1.
        % Процитированные здесь работы:
        %  * подсчитываются, для автоматического составления фразы "Основные результаты ..."
        %  * попадают в авторскую библиографию, при usefootcite==0 и стиле `\insertbiblioauthor` или `\insertbiblioauthorgrouped`
        %  * нумеруются там в зависимости от порядка команд `\printbibliography` в этом разделе.
        %  * при использовании `\insertbiblioauthorgrouped`, порядок команд `\printbibliography` в нем должен быть тем же (см. biblio/biblatex.tex)
        %
        % Невидимый библиографический список для подсчета количества публикаций:
        \printbibliography[heading=nobibheading, section=1, env=countauthorvak,          keyword=biblioauthorvak]%
        \printbibliography[heading=nobibheading, section=1, env=countauthorwos,          keyword=biblioauthorwos]%
        \printbibliography[heading=nobibheading, section=1, env=countauthorscopus,       keyword=biblioauthorscopus]%
        \printbibliography[heading=nobibheading, section=1, env=countauthorconf,         keyword=biblioauthorconf]%
        \printbibliography[heading=nobibheading, section=1, env=countauthorother,        keyword=biblioauthorother]%
        \printbibliography[heading=nobibheading, section=1, env=countregistered,         keyword=biblioregistered]%
        \printbibliography[heading=nobibheading, section=1, env=countauthorpatent,       keyword=biblioauthorpatent]%
        \printbibliography[heading=nobibheading, section=1, env=countauthorprogram,      keyword=biblioauthorprogram]%
        \printbibliography[heading=nobibheading, section=1, env=countauthor,             keyword=biblioauthor]%
        \printbibliography[heading=nobibheading, section=1, env=countauthorvakscopuswos, filter=vakscopuswos]%
        \printbibliography[heading=nobibheading, section=1, env=countauthorscopuswos,    filter=scopuswos]%
        %
        \nocite{*}%
        %
        {\publications} Основные результаты по теме диссертации изложены в~\arabic{citeauthor}~печатных изданиях,
        \arabic{citeauthorvak} из которых изданы в журналах, рекомендованных ВАК\sloppy%
        \ifnum \value{citeauthorscopuswos}>0%
            , \arabic{citeauthorscopuswos} "--- в~периодических научных журналах, индексируемых Web of~Science и Scopus\sloppy%
        \fi%
        \ifnum \value{citeauthorconf}>0%
            , \arabic{citeauthorconf} "--- в~тезисах докладов.
        \else%
            .
        \fi%
        \ifnum \value{citeregistered}=1%
            \ifnum \value{citeauthorpatent}=1%
                Зарегистрирован \arabic{citeauthorpatent} патент.
            \fi%
            \ifnum \value{citeauthorprogram}=1%
                Зарегистрирована \arabic{citeauthorprogram} программа для ЭВМ.
            \fi%
        \fi%
        \ifnum \value{citeregistered}>1%
            Зарегистрированы\ %
            \ifnum \value{citeauthorpatent}>0%
            \formbytotal{citeauthorpatent}{патент}{}{а}{}\sloppy%
            \ifnum \value{citeauthorprogram}=0 . \else \ и~\fi%
            \fi%
            \ifnum \value{citeauthorprogram}>0%
            \formbytotal{citeauthorprogram}{программ}{а}{ы}{} для ЭВМ.
            \fi%
        \fi%
        % К публикациям, в которых излагаются основные научные результаты диссертации на соискание ученой
        % степени, в рецензируемых изданиях приравниваются патенты на изобретения, патенты (свидетельства) на
        % полезную модель, патенты на промышленный образец, патенты на селекционные достижения, свидетельства
        % на программу для электронных вычислительных машин, базу данных, топологию интегральных микросхем,
        % зарегистрированные в установленном порядке.(в ред. Постановления Правительства РФ от 21.04.2016 N 335)
    \end{refsection}%
    \begin{refsection}[bl-author, bl-registered]
        % Это refsection=2.
        % Процитированные здесь работы:
        %  * попадают в авторскую библиографию, при usefootcite==0 и стиле `\insertbiblioauthorimportant`.
        %  * ни на что не влияют в противном случае
        %\nocite{vakbib2}%vak
        %\nocite{patbib1}%patent
        \nocite{progbib1}%program
        \nocite{bib1}%other
        \nocite{bib2}%other
        \nocite{confbib1}%conf
    \end{refsection}%
        %
        % Все, что вне этих двух refsection, это refsection=0,
        %  * для диссертации - это нормальные ссылки, попадающие в обычную библиографию
        %  * для автореферата:
        %     * при usefootcite==0, ссылка корректно сработает только для источника из `external.bib`. Для своих работ --- напечатает "[0]" (и даже Warning не вылезет).
        %     * при usefootcite==1, ссылка сработает нормально. В авторской библиографии будут только процитированные в refsection=0 работы.
}

% При использовании пакета \verb!biblatex! будут подсчитаны все работы, добавленные
% в файл \verb!biblio/author.bib!. Для правильного подсчета работ в~различных
% системах цитирования требуется использовать поля:
% \begin{itemize}
%         \item \texttt{authorvak} если публикация индексирована ВАК,
%         \item \texttt{authorscopus} если публикация индексирована Scopus,
%         \item \texttt{authorwos} если публикация индексирована Web of Science,
%         \item \texttt{authorconf} для докладов конференций,
%         \item \texttt{authorpatent} для патентов,
%         \item \texttt{authorprogram} для зарегистрированных программ для ЭВМ,
%         \item \texttt{authorother} для других публикаций.
% \end{itemize}
% Для подсчета используются счетчики:
% \begin{itemize}
%         \item \texttt{citeauthorvak} для работ, индексируемых ВАК,
%         \item \texttt{citeauthorscopus} для работ, индексируемых Scopus,
%         \item \texttt{citeauthorwos} для работ, индексируемых Web of Science,
%         \item \texttt{citeauthorvakscopuswos} для работ, индексируемых одной из трех баз,
%         \item \texttt{citeauthorscopuswos} для работ, индексируемых Scopus или Web of~Science,
%         \item \texttt{citeauthorconf} для докладов на конференциях,
%         \item \texttt{citeauthorother} для остальных работ,
%         \item \texttt{citeauthorpatent} для патентов,
%         \item \texttt{citeauthorprogram} для зарегистрированных программ для ЭВМ,
%         \item \texttt{citeauthor} для суммарного количества работ.
% \end{itemize}
% % Счетчик \texttt{citeexternal} используется для подсчета процитированных публикаций;
% % \texttt{citeregistered} "--- для подсчета суммарного количества патентов и программ для ЭВМ.

% Для добавления в список публикаций автора работ, которые не были процитированы в
% автореферате, требуется их~перечислить с использованием команды \verb!\nocite! в
% \verb!Synopsis/content.tex!.