%%% Основные сведения %%%
\newcommand{\thesisAuthorLastName}{Кривоносов}
\newcommand{\thesisAuthorOtherNames}{Михаил Игоревич}
\newcommand{\thesisAuthorInitials}{М.\,И.}
\newcommand{\thesisAuthor}             % Диссертация, ФИО автора
{%
    \texorpdfstring{% \texorpdfstring takes two arguments and uses the first for (La)TeX and the second for pdf
        \thesisAuthorLastName~\thesisAuthorOtherNames% так будет отображаться на титульном листе или в тексте, где будет использоваться переменная
    }{%
        \thesisAuthorLastName, \thesisAuthorOtherNames% эта запись для свойств pdf-файла. В таком виде, если pdf будет обработан программами для сбора библиографических сведений, будет правильно представлена фамилия.
    }
}
\newcommand{\thesisAuthorShort}        % Диссертация, ФИО автора инициалами
{\thesisAuthorInitials~\thesisAuthorLastName}
%\newcommand{\thesisUdk}                % Диссертация, УДК
%{xxx.xxx}
\newcommand{\thesisTitle}              % Диссертация, название
{Случайные блуждания как стратегии поиска: моделирование и оценка эффективности}
\newcommand{\thesisSpecialtyNumber}    % Диссертация, специальность, номер
{1.3.4.}
\newcommand{\thesisSpecialtyTitle}     % Диссертация, специальность, название (название взято с сайта ВАК для примера)
{Радиофизика}
%% \newcommand{\thesisSpecialtyTwoNumber} % Диссертация, вторая специальность, номер
%% {XX.XX.XX}
%% \newcommand{\thesisSpecialtyTwoTitle}  % Диссертация, вторая специальность, название
%% {\fixme{Теория и~методика физического воспитания, спортивной тренировки,
%% оздоровительной и~адаптивной физической культуры}}
\newcommand{\thesisDegree}             % Диссертация, ученая степень
{кандидата физико-математических наук}
\newcommand{\thesisDegreeShort}        % Диссертация, ученая степень, краткая запись
{канд. физ.-мат. наук}
\newcommand{\thesisCity}               % Диссертация, город написания диссертации
{Нижний Новгород}
\newcommand{\thesisYear}               % Диссертация, год написания диссертации
{\the\year}
\newcommand{\thesisOrganization}       % Диссертация, организация
{Федеральное государственное автономное образовательное учреждение высшего образования «Национальный исследовательский Нижегородский государственный университет им. Н.И. Лобачевского» (ННГУ)}
\newcommand{\thesisOrganizationShort}  % Диссертация, краткое название организации для доклада
{ННГУ им. Н.И. Лобачевского}

\newcommand{\thesisInOrganization}     % Диссертация, организация в предложном падеже: Работа выполнена в ...
{ФГАОУ ВО «Национальный исследовательский Нижегородский государственный университет им. Н.И. Лобачевского» (ННГУ)}

%% \newcommand{\supervisorDead}{}           % Рисовать рамку вокруг фамилии
\newcommand{\supervisorFio}              % Научный руководитель, ФИО
{Иванченко Михаил Васильевич}
\newcommand{\supervisorRegalia}          % Научный руководитель, регалии
{доктор физико-математических наук, доцент, ИИТММ, ННГУ им.~Н.И.~Лобачевского}
\newcommand{\supervisorFioShort}         % Научный руководитель, ФИО
{М.\,В.~Иванченко}
\newcommand{\supervisorRegaliaShort}     % Научный руководитель, регалии
{д.ф.-м.н.,доцент}

%% \newcommand{\supervisorTwoDead}{}        % Рисовать рамку вокруг фамилии
%% \newcommand{\supervisorTwoFio}           % Второй научный руководитель, ФИО
%% {Фамилия Имя Отчество}
%% \newcommand{\supervisorTwoRegalia}       % Второй научный руководитель, регалии
%% {уч. степень, уч. звание}
%% \newcommand{\supervisorTwoFioShort}      % Второй научный руководитель, ФИО
%% {И.\,О.~Фамилия}
%% \newcommand{\supervisorTwoRegaliaShort}  % Второй научный руководитель, регалии
%% {уч.~ст.,~уч.~зв.}

\newcommand{\opponentOneFio}           % Оппонент 1, ФИО
{Постников Евгений Борисович}
\newcommand{\opponentOneRegalia}       % Оппонент 1, регалии
{доктор физико-математических наук}
\newcommand{\opponentOneJobPlace}      % Оппонент 1, место работы
{Курский государственный университет}
\newcommand{\opponentOneJobPost}       % Оппонент 1, должность
{профессор отдела теоретической физики научно-исследовательского центра физики конденсированного состояния}

\newcommand{\opponentTwoFio}           % Оппонент 2, ФИО
{Андреев Андрей Викторович}
\newcommand{\opponentTwoRegalia}       % Оппонент 2, регалии
{кандидат физико-математических наук}
\newcommand{\opponentTwoJobPlace}      % Оппонент 2, место работы
{Балтийский федеральный университет имени Иммануила Канта}
\newcommand{\opponentTwoJobPost}       % Оппонент 2, должность
{старший научный сотрудник центра нейротехнологий и машинного обучения}

%% \newcommand{\opponentThreeFio}         % Оппонент 3, ФИО
%% {\fixme{Фамилия Имя Отчество}}
%% \newcommand{\opponentThreeRegalia}     % Оппонент 3, регалии
%% {\fixme{кандидат физико-математических наук}}
%% \newcommand{\opponentThreeJobPlace}    % Оппонент 3, место работы
%% {\fixme{Основное место работы c длинным длинным длинным длинным названием}}
%% \newcommand{\opponentThreeJobPost}     % Оппонент 3, должность
%% {\fixme{старший научный сотрудник}}







\newcommand{\leadingOrganizationTitle} % Ведущая организация, дополнительные строки. Удалить, чтобы не отображать в автореферате
{Федеральное государственное бюджетное образовательное учреждение высшего образования <<Саратовский национальный исследовательский государственный университет имени Н. Г. Чернышевского>>}

\newcommand{\defenseDate}              % Защита, дата
{\fixme{DD mmmmmmmm YYYY~г.~в~XX часов}}
\newcommand{\defenseCouncilNumber}     % Защита, номер диссертационного совета
{24.2.340.03}
\newcommand{\defenseCouncilTitle}      % Защита, учреждение диссертационного совета
{Нижегородском государственном университете им. Н.И. Лобачевского}
\newcommand{\defenseCouncilAddress}    % Защита, адрес учреждение диссертационного совета
{603022, г. Нижний Новгород, пр. Гагарина, 23, корп. 1, ауд. 420.}
\newcommand{\defenseCouncilPhone}      % Телефон для справок
{\fixme{+7~(0000)~00-00-00}}

\newcommand{\defenseSecretaryFio}      % Секретарь диссертационного совета, ФИО
{Клюев Алексей Викторович}
\newcommand{\defenseSecretaryRegalia}  % Секретарь диссертационного совета, регалии
{д.~ф.-м.~н.}            % Для сокращений есть ГОСТы, например: ГОСТ Р 7.0.12-2011 + http://base.garant.ru/179724/#block_30000

\newcommand{\synopsisLibrary}          % Автореферат, название библиотеки
{Фундаментальной библиотеке Нижегородского государственного университета им. Н.И. Лобачевского и на сайте:}
\newcommand{\synopsisDate}             % Автореферат, дата рассылки
{\fixme{DD mmmmmmmm}\the\year~года}

% To avoid conflict with beamer class use \providecommand
\providecommand{\keywords}%            % Ключевые слова для метаданных PDF диссертации и автореферата
{}
